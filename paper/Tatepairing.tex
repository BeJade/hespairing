\section{The reduced Tate pairing}\label{sec:Tate}

Let $E$ be an elliptic curve defined over a finite field $\FF_{p}$ where $p$ is a prime, let $r$ be the size of the largest prime-order subgroup of $E(\FF_{p})$, and let $\mu_{r} \subseteq \FF_{p^{k}}^{*}$ be the group of $r^{\text{th}}$ roots of unity. We define the \emph{embedding degree} of $E$ to be
\[\text{min}\{k \in \mathbb{Z}: r|p^{k}-1\}.\]
For an integer $0<e\leq k$ and a point $T \in E(\FF_{p^{e}})$, we define the function $f_{r,T}$ to be the unique normalized function with divisor
\[\text{div}(f_{r,T}) = r[T]-[rT]-(r-1)[T_{\infty}],\] where $T_{\infty}$ denotes the point at infinity on $E$.
\begin{definition}
With notation as above, we define the \emph{reduced Tate pairing on $E$} to be
\[\begin{array}{rccc}
\tau_{r} : & E(\FF_{p^{k}})[r] \times E(\FF_{p^{k}})/[r]E(\FF_{p^{k}}) & \longrightarrow & \mu_{r} \\
& (P,Q) & \mapsto & f_{r,P}(Q)^{\frac{p^{k}-1}{r}}.
\end{array}\]
\end{definition}
In this paper we address the computation of the reduced Tate pairing restricted to $G_{1} \times G_{2}$, where
\[G_{1} = E[r] \cap \ker(\phi_{p}-[1])\] and
\[G_{2} = E[r] \cap \ker(\phi_{p} - [p]) \subseteq E(\FF_{p^{k}}).\] Here $\phi_{p}$ denotes the $p$-power Frobenius morphism on $E$.
We denote the restriction of $\tau_{r}$ to $G_{1} \times G_{2}$ by
\[e_{r} : G_{1} \times G_{2} \longrightarrow \mu_{r}.\]
It is easy to check on a case-by-case basis that this pairing is non-degenerate and bilinear. (Non-degeneracy and bilinearity are automatic for ordinary elliptic curves, but the curves that we consider are supersingular). The following facts, due to Miller (\cite{Mil}), allow us to explicitly compute this pairing. Let $0<e \leq k$ be an integer, and let $T,T' \in E(\FF_{q^{e}})$. We will write $L_{T,T'}$ for the projective line passing through $T$ and $T'$, and we will write $g_{T,T'}$ for the function
\[g_{T,T'}= \frac{L_{T,T'}}{L_{T+T',-(T+T')}}.\]

\begin{lemma}
With notation as above, for $P\in G_{1}$, we have that $f_{0,P} = f_{1,P}=1$ and for $n \geq 2$, that $f_{n+1,P} = f_{n,P}g_{P,nP}$.
\end{lemma}

\begin{proof}
Straightforward computation on divisors. See \cite{Mil} for more details.
\end{proof}

\begin{corollary}
With notation as above, for $P \in G_{1}$ and $Q \in G_{2}$, we get that
\[e_{r}(P,Q)= \prod_{m=1}^{r-1} \frac{L_{P,mP}(Q)^{\frac{p^{k}-1}{r}}}{L_{(m+1)P,-(m+1)P}(Q)^{\frac{p^{k}-1}{r}}}.\]
\end{corollary}

Hence, we only need to give formulas for addition and point doubling to compute the reduced Tate pairing restricted to $G_{1} \times G_{2}$. That is, Algorithm \ref{algo:miller} computes $e_{r}(P,Q)$ for $(P,Q) \in G_{1} \times G_{2}$. For a point $R$ on $E$, we denote by $\ell_{2R}$ the tangent line at $R$ and by $\ell_{R,P}$ the line passing through $R$ and $P$.

\begin{algorithm}
\caption{Miller's algorithm}
\label{algo:miller}
\begin{algorithmic}[1]

%	\Require $R = P$ and $f = 1$

	\For {$i := \ell - 2$ {\bf{down to}} $0$}
		\State $f \leftarrow f^2 \cdot \ell_{2R}(Q)$
		\State $R \leftarrow 2R$
		\If {$m_i = 1$}
			\State $f \leftarrow f \cdot \ell_{R,P}(Q)$
			\State $R \leftarrow R+P$
		\EndIf
	\EndFor
	\State $f \leftarrow f^{(p^k-1)/r}$

\end{algorithmic}
\end{algorithm}
