\section{Explicit formulas}
\label{sec:formulas}

Recall the twisted Hessian curve equation in projective coordinates
%$ a x^3 + y^3 = 1 + d x y, $ or
$$ \mathcal{H}: a X^3 + Y^3 + Z^3 = d X Y Z. $$
This section shows formulas for computing point doubling, point addition, and line functions associated to these operations.
These formulas work under an assumption that the curve parameter $d = 0$.

We denote operations in the base field, namely, 
field multiplication, field squaring and field multiplication by curve constant $a$
by $\mul, \sqr, \mulp{a}$ respectively.
We denote embedding degree and twist degree by $k$ and $d$.

The following formulas work for both Tate and ate pairings.
We denote
$P = (X_1,Y_1,1)$ and $R = (X_2,Y_2,Z_2)$ on $E/\F{p}$ and a point $Q = (X_Q,Y_Q,1)$ on $E/\F{p^{k}}$
Note the points $P$ and $Q$ do not change throughout the pairing computation and thus can be preprocessed,
namely, we assume that the $P_Z = Q_Z = 1$.

%We also follow~\cite{2010/CLN} to assume that pairings can be computed entirely on twisted curves.
%Therefore, we denote by $k/3$.

%%%%%%%%%%%%%%%%%%%%%%%%%%%%%%%%%%%%%%%%%%%%%%%%%%%%%%%%%%%%%%%%%%%%%%%%%%%%%%%%

\subsection{Doubling}
First, we show formulas for compute point doubling.
These formulas work under the condition that the curve parameter $d=0$,
which is the case for pairing-friendly twisted Hessian given by our constructions.
%Given a point $P = (X_1,Y_1,Z_1)$ on $E/\F{p}$,
%and a point $Q = (X_Q,Y_Q,1)$ on $E/\F{p^{k}}$
%where $X_Q \in \F{p^{k}}$ and $Y_Q \in \F{p^{k/d}}$,
Let $R = (X_1, Y_1, Z_1)$.
The following formulas compute the doubling of point $R = 2R = (X_3,Y_3,Z_3)$.
%and the line function $l$.
\begin{align*}
T &= Y_1^2;\	\qquad	A = Y_1 \cdot T;\	\qquad
S = Z_1 ^ 2;\	\qquad	B = Z_1 \cdot S;\\
X_3 &= X_1 \cdot (A - B);\	\quad
Y_3 = -Z_1 \cdot (2A + B);\	\quad
Z_3 = Y_1 \cdot (A + 2B).
\end{align*}
To compute point doubling using these formulas,
it costs $5\mul + 2\sqr + 2\mulp{2} + 3\add$ plus one negation.

Let $Q = (X_Q, Y_Q, 1)$ be the point to evaluate.
Let $R$ and $2R$ be as defined above for point doubling.
The associated line function for doubling
to compute $l_{2R}(Q)$ is as follows:
\begin{align*}
l_1 &= a X_1^2 X_Q + T \cdot Y_Q + S;\	\qquad
l_a = X_3 \cdot Y_Q + X_3;	\\
l_b &= X_Q \cdot (Y_3 + Z_3);\			\qquad
%l_c = (l_a^2 + l_a \cdot l_b + l_b^2);\		\qquad
l_c = l_a^2 + l_b \cdot (l_a + l_b);\		\qquad
l = l_1 \cdot l_c.
\end{align*}
To compute the line function for doubling,
it costs 
$k(\frac{14}{3}\mul + \frac{1}{3}\sqr) + \sqr + 1\mulp{a}$
Therefore, the total number of operations for computing ``doubling step'' in the pairing computation is
$k(\frac{14}{3}\mul + \frac{1}{3}\sqr) + 5\mul +3\sqr + 1\mulp{a} + 2\mulp{2} + 3\add$.

%%$ k(2\mul + \frac{2}{3}\sqr) + 5\mul +3\sqr + 1\mul_a$ for {\it{odd}} embedding degrees.
%$ k(\frac{14}{3}\mul + \frac{1}{3}\sqr) + 5\mul +3\sqr + 1\mul_a$.
%%Note that curves generated by our Construction 1 have even embedding degrees.
%%The total number of operations for {\it{even}} embedding degrees is
%%$ \frac{4}{3}k\mul + 5\mul +3\sqr + 1\mul_a$.

%%%%%%%%%%%%%%%%%%%%%%%%%%%%%%%%%%%%%%%%%%%%%%%%%%%%%%%%%%%%%%%%%%%%%%%%%%%%%%%%

\subsection{Addition}
Let $P = (X_1,Y_1,1)$ and $R = (X_2,Y_2,Z_2)$.
The following formulas compute the addition of points
$P + R = (X_1,Y_1,1) + (X_2,Y_2,Z_2) = (X_3,Y_3,Z_3)$.
%Given points $P = (X_1,Y_1,1)$ and $R = (X_2,Y_2,Z_2)$ on $E/\F{p}$ and a point $Q = (X_Q,Y_Q,1)$ on $E/\F{p^{k}}$
%where $X_Q \in \F{p^{k}}$ and $Y_Q \in \F{p^{k/d}}$,
%the following formulas compute the addition of points $P + R = (X_1,Y_1,1) + (X_2,Y_2,Z_2) = (X_3,Y_3,Z_3)$ and the line function $l$.
\begin{align*}
A &= X_1 \cdot Z_2;\	\qquad
C = Y_1 \cdot X_2;\	\qquad
D = Y_1 \cdot Y_2;\	\qquad
F = a X_1 \cdot X_2;\\
G &= (D + Z_2) \cdot (A - C);\	\quad
H = (D - Z_2) \cdot (A + C);\\
J &= (D + F) \cdot (A - Y_2);\	\quad
K = (D - F) \cdot (A + Y_2);\\
X_3 &= G - H;\	\qquad
Y_3 = K - J;\\
Z_3 &= J + K - G - H - 2(Z_2 - F) \cdot (C + Y_2).
\end{align*}
To compute point addition using these formulas,
it costs $9\mul + \mulp{a} + \mulp{2} + 16\add$.

Let $Q = (X_Q,Y_Q,1)$ be the point to evaluate.
Let $P,R,(P+R)$ be as defined above for point addition.
The associated line function for addition
to compute $l_{R,P}$ is as follows:
\begin{align*}
l_1 &= (Y_1 \cdot Z_2 - Y_2) \cdot (X_1 - X_Q) + (Y_Q - Y_1) \cdot (X_1 \cdot Z_2 - X_2);\\
l_a &= Y_Q \cdot X_3 + X_3;\	\quad
l_b = X_Q \cdot (Y_3 + Z_3);\	\quad
%l_c = (l_a^2 + l_a \cdot l_b + l_b^2);\\
l_c = l_a^2 + l_a \cdot (l_b + l_b);\\
l &= l_1 \cdot \l_c.
\end{align*}
To compute the line function for addition,
it costs
$k(\frac{14}{3}\mul + \frac{1}{3}\sqr) + 2\mul + 9\add$.
Therefore, the total number of operations to compute ``addition step'' in the pairing computation is
$k(\frac{14}{3}\mul + \frac{1}{3}\sqr) + 11\mul + \mulp{a} + \mulp{2} + 25\add$.


%$ k(\frac{14}{3}\mul + \frac{1}{3}\sqr) + 11\mul + 1\mul_a$.
%%$ \frac{4}{3}k\mul + 11\mul + 1\mul_a$ for {\it{even}} embedding degrees.


