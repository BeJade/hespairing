\section{Explicit formulas}
\label{sec:formulas}

Recall the twisted Hessian curve equation in projective coordinates
$$ \mathcal{H}: a' X^3 + Y^3 + Z^3 = d' X Y Z. $$
This section shows formulas for computing point doubling, point addition,
and line functions associated to these operations.
These formulas work under an assumption that the curve parameter $d' = 0$.

We denote operations in the base field, namely, 
field multiplication, field squaring and field multiplication by curve constant $a$
by $\mul, \sqr, \mulp{a}$ respectively.
We denote the embedding degree and the twist degree by $k$ and $d=3$.

The following formulas work for both Tate and ate pairings.
We denote a point $P = (X_1,Y_1,1)$ and a point $R = (X_2,Y_2,Z_2)$ on $E/\F{q}$
and a point $Q' = (X_{Q'},Y_{Q'},1)$ on $E/\F{q^{k/d}}$.
Note the points $P$ and $Q'$ do not change throughout the pairing computation and thus can be preprocessed,
namely, we assume that the $X_1 = Z_{Q'} = 1$.

%We also follow~\cite{2010/CLN} to assume that pairings can be computed entirely on twisted curves.
%Therefore, we denote by $k/3$.

%%%%%%%%%%%%%%%%%%%%%%%%%%%%%%%%%%%%%%%%%%%%%%%%%%%%%%%%%%%%%%%%%%%%%%%%%%%%%%%%

\subsection{Doubling}
First, we show formulas for compute point doubling.
These formulas work under the condition that the curve parameter $d'=0$,
which is the case for pairing-friendly twisted Hessian generated by our constructions.
Let $R = (X_1, Y_1, Z_1)$.
The following formulas compute the doubling of point $R = 2R = (X_3,Y_3,Z_3)$.
\begin{align*}
T &= Y_1^2;\	\qquad	A = Y_1 \cdot T;\	\qquad
S = Z_1 ^ 2;\	\qquad	B = Z_1 \cdot S;\\
X_3 &= X_1 \cdot (A - B);\	\quad
Y_3 = -Z_1 \cdot (2A + B);\	\quad
Z_3 = Y_1 \cdot (A + 2B).
\end{align*}
To compute point doubling using these formulas,
it costs $5\mul + 2\sqr$. % + 2\mulp{2} + 3\add$ plus one negation.

Let $Q' = (X_{Q'}, Y_{Q'}, 1)$ be the point to evaluate.
Let $R$ and $2R$ be as defined above for point doubling.
The associated line function for doubling
to compute $l_{2R}(Q')$ is as follows
\begin{align*}
l_1 &= a X_1^2 X_{Q'} + T \cdot Y_{Q'} + S;\		\qquad
l_a = X_3 \cdot Y_{Q'} + X_3;			\\
l_b &= X_{Q'} \cdot (Y_3 + Z_3);\			\qquad
l_c = l_a^2 + l_b \cdot (l_a + l_b);\		\qquad
l = l_1 \cdot l_c.
\end{align*}
To compute the line function for doubling,
it costs 
$\frac{k}{3}(6\mul + \sqr) + \sqr + \mulp{a}$. % + 6\add$
Therefore, the total number of operations for computing ``doubling step''
({line 3} in Algorithm~\ref{algo:miller}) in the pairing computation is
$\frac{k}{3}(6\mul + \sqr) + 5\mul +3\sqr + \mulp{a}$. % + 2\mulp{2} + 9\add$.

%$ k(\frac{14}{3}\mul + \frac{1}{3}\sqr) + 5\mul +3\sqr + 1\mul_a$.

%%%%%%%%%%%%%%%%%%%%%%%%%%%%%%%%%%%%%%%%%%%%%%%%%%%%%%%%%%%%%%%%%%%%%%%%%%%%%%%%

\subsection{Addition}
Let $P = (X_1,Y_1,1)$ and $R = (X_2,Y_2,Z_2)$.
The following formulas compute the addition of points
$P + R = (X_1,Y_1,1) + (X_2,Y_2,Z_2) = (X_3,Y_3,Z_3)$.
\begin{align*}
A &= X_1 \cdot Z_2;\	\qquad
C = Y_1 \cdot X_2;\	\qquad
D = Y_1 \cdot Y_2;\	\qquad
F = a X_1 \cdot X_2;\\
G &= (D + Z_2) \cdot (A - C);\	\quad
H = (D - Z_2) \cdot (A + C);\\
J &= (D + F) \cdot (A - Y_2);\	\quad
K = (D - F) \cdot (A + Y_2);\\
X_3 &= G - H;\	\qquad
Y_3 = K - J;\\
Z_3 &= J + K - G - H - 2(Z_2 - F) \cdot (C + Y_2).
\end{align*}
To compute point addition using these formulas,
it costs $9\mul + \mulp{a}$. % + \mulp{2} + 16\add$.

Let $Q' = (X_{Q'},Y_{Q'},1)$ be the point to evaluate.
Let $P,R,(P+R)$ be as defined above for point addition.
The associated line function for addition
to compute $l_{R,P}$ is as follows
\begin{align*}
l_1 &= (Y_1 \cdot Z_2 - Y_2) \cdot (X_1 - X_{Q'}) + (Y_{Q'} - Y_1) \cdot (X_1 \cdot Z_2 - X_2);\\
l_a &= Y_{Q'} \cdot X_3 + X_3;\	\quad
l_b = X_{Q'} \cdot (Y_3 + Z_3);\	\quad
l_c = l_a^2 + l_a \cdot (l_b + l_b);\\
l &= l_1 \cdot \l_c.
\end{align*}
To compute the line function for addition,
it costs
$\frac{k}{3}(6\mul + \sqr) + 2\mul$. % + 9\add$.
Therefore, the total number of operations to compute ``addition step''
({line 6} in Algorithm~\ref{algo:miller}) in the pairing computation is
$\frac{k}{3}(6\mul + \sqr) + 11\mul + \mulp{a}$. % + \mulp{2} + 25\add$.

%$ k(\frac{14}{3}\mul + \frac{1}{3}\sqr) + 11\mul + 1\mul_a$.


