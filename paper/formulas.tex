\section{Computation of line functions}

We now move to the explicit computation of the line functions $\ell_{1} = L_{P,Q}$ and $\ell_{2} = L_{P,-P}$ given in the formula for the reduced Tate pairing. Recall from \ref{?} that the denominator of the reduced Tate pairing was given by
\[L_{nP,-nP}(Q)^{\frac{q^{k}-1}{r}},\] where $P \in E(\FF_{q})$ and $Q \in E(\FF_{q^{k}})$. Recall also from \ref{?} that, possibly replacing $E$ by a twist, we may assume without loss of generality that $Q \in E(\FF_{q^{k/3}})$. Write $nP = (x_{nP},y_{nP},z_{nP})$ and $Q=(x_{Q},y_{Q},z_{Q})$. Then applying the group law we compute the equation of $L_{nP,-nP}$ to be
\[Y + Z + (y_{nP} + z_{nP})X = 0,\] which evaluated at $Q$ gives an element $b$ of $\FF_{q^{k/3}}$. Now observe that
\[(q^{k} -1) = (q^{k/3}-1)(\sum_{i=0}^{2k/3} q^{i}),\] and by minimality of the embedding degree that $r \not | q^{k/3}-1$, hence
\[L_{nP,-nP}(Q)^{\frac{q^{k}-1}{r}} = b^{\frac{q^{k}-1}{r}} = (b^{q^{k/3}-1})^\frac{\sum_{i=0}^{2k/3} q^{i}}{r} = 1.\] So this does not contribute to the reduced Tate pairing.


%%%%%%%%%%%%%%%%%%%%%%%%%%%%%%%%%%%%%%%%%%%%%%%%%%%%%%%%%%%%%%%%%%%%%%%%%%%%%%%%%%%%%%%%%%%%%%%%%%%%

\subsection{Denominator elimination}
We follow a similar technique as in~\cite{2008/lin} and~\cite{2011/mrabet}
by rewriting an inversion into a fraction having denominator lies in a subfield.
Define a finite field extension $\F{q^m}$ of a finite field $\F{q}$.
%Define a finite field extension $F = \F{q^m}$ of a finite field $K = \F{q}$.
%For $\alpha \in F = \F{q^m}$ and $K = \F{q}$,
%the \emph{norm} $N_{F/K}(\alpha)$ of $\alpha$ over $K$ is defined by
%$$ N_{F/K}(\alpha) = \alpha \cdot \alpha^q \cdot \cdots \cdot \alpha^{q^{m-1}} = \alpha^{(q^m-1)/(q-1)} $$
%
%one of properties
%$N_{F/K}$ maps $F$ onto $K$ and $F^\ast$ on $K^\ast$.
%Let $q = p^7$.
Observe that
$$ \frac{1}{a} = \frac{a^q \cdot a^{q^2}} {a \cdot a^{q} \cdot a^{q^2}} $$
where $a \cdot a^{q} \cdot a^{q^2} = N(a)$, i.e., a $norm$ of $a$ which lies in $\F{q}$.
Because $N(a) \in \F{q}$ which is a subfield of $\F{q^m}$,
the final exponentiation forces $N(a)$ to become 1.
This means that we can ignore the computation of the denominator $N(a)$.
Therefore, instead of computing $\frac{1}{a}$,
we compute $a^{q} \cdot a^{q^2}$ where
$a^{q}$ and $a^{q^2}$ can be easily computed using Frobenius.

%%%%%%%%%%%%%%%%%%%%%%%%%%%%%%%%%%%%%%%%%%%%%%%%%%%%%%%%%%%%%%%%%%%%%%%%%%%%%%%%%%%%%%%%%%%%%%%%%%%%

\subsection{Explicit formulas}

Recall the twsited Hessian curve equation in projective coordiates
%$ a x^3 + y^3 = 1 + d x y, $ or
$$ \mathcal{H}: a X^3 + Y^3 = Z^3 + d X Y Z. $$
This subsection shows formulas to compute point doubling, point addition, and line functions associated to these operations.
These formulas work under an assumption that the curve parameter $d = 0$.

Symbols $\mul, \sqr, \add, \mulp{a}, \mulp{2}$ denote
field multiplication, field squaring, field addition,
field multiplication by curve constant $a$ and field multiplication by 2 over $\F{p}$,
whereas $\mul_k$ and $\add_k$ denote
field multiplication and field addition over $\F{p^k}$,
and $\mul_e$ and $\add_e$ denote
field multiplication and field addition over $\F{p^e}$ where $e = k/d$.

\subsubsection{Doubling.}
%	// 5M + 2S + 3*3 + 1*2 + 1*neg + 6*add
%	// assume d = 0
Given a point $P = (X_1,Y_1,Z_1)$ on $E/\F{p}$ and a point $Q = (X_Q,Y_Q,1)$ on $E'/\F{p^{k/d}}$,
the following formulas compute the doubling of point $P = 2P = (X_3,Y_3,Z_3)$ and the line function $l$.
\begin{align*}
U &= Y_1^2;\	\qquad	A = Y_1 \cdot U;\	\qquad
V = Z_1 ^ 2;\	\qquad	B = Z_1 \cdot V;\\
X_3 &= X_1 \cdot (A - B);\	\quad
Y_3 = -Z_1 \cdot (2A + B);\	\quad
Z_3 = Y_1 \cdot (A + 2B);\\
l_1 &= -X_Q \cdot (A + B) + X_1 \cdot (Y_Q \cdot U + V);	\\
l_2 &= X_3 \cdot (Y_Q + 1) - X_Q \cdot (Y_3 + Z_3);	\\
l &= l_1 \cdot Frobenius(l_2,7) \cdot Frobenius(l_2,14);
\end{align*}
The total number of operations is
$ 5\mul +2\sqr + 3\add + 2\mulp{2} + $
%$ 1\neg+ $
$ (2\mul_k + 3\mul_e + 2\add_k + 2\add_e + 2\add) + $
$ (2\mul_k + 2Frobenius$).

\subsubsection{Addition.}
Given points $P = (X_1,Y_1,1)$ and $R = (X_2,Y_2,Z_2)$ on $E/\F{p}$ and a point $Q = (X_Q,Y_Q,1)$ on $E'/\F{p^{k/d}}$,
the following formulas compute the addition of points $P + R = (X_1,Y_1,Z_1) + (X_2,Y_2,Z_2) = (X_3,Y_3,Z_3)$ and the line function $l$.
\begin{align*}
A &= X_1 \cdot Z_2;\	\qquad
C = Y_1 \cdot X_2;\	\qquad
D = Y_1 \cdot Y_2;\	\qquad
F = a X_1 \cdot X_2;\\
G &= (D + Z_2) \cdot (A - C);\	\quad
H = (D - Z_2) \cdot (A + C);\\
J &= (D + F) \cdot (A - Y_2);\	\quad
K = (D - F) \cdot (A + Y_2);\\
X_3 &= G - H;\	\qquad
Y_3 = K - J;\\
Z_3 &= J + K - G - H - 2(B - F) \cdot (C + E);\\
l_1 &= (Y_1 \cdot Z_2 - Y_2) \cdot (X_1 - X_Q) + (Y_Q - Y_1) \cdot (X_1 \cdot Z_2 - X_2);\\
l_2 &= X_3 \cdot (Y_Q + 1) - X_Q \cdot (Y_3 + Z_3);\\
l &= l_1 \cdot Frobenius(l_2,7) \cdot Frobenius(l_2,14);
\end{align*}
The total number of operations is
$ 9\mul +16\add + 1\mulp{a} + 1\mulp{2} + $
$ (2\mul_k + 2\mul_e + 2\mul + 3\add_k + 2\add_e + 3\add) + $
$ (2\mul_k + 2Frobenius)$.

