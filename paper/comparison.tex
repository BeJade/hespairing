\section{Comparison}
\label{sec:cmp}

We would like to emphasize that:
\begin{itemize}
\item	Most of the previous works concerned Weierstrass curves.
\item	Most of the previous works concerned even embedding degrees.
\item	There were some previous works on Hessian curves but with even embedding degrees.
\item	There were also some previous works with odd embedding degrees but using Weierstrass curves.
\end{itemize}
Therefore, it is difficult to have an appropriate comparison between this work and previous works
since this work happens to be the first to consider the twisted Hessian curves with odd embedding degrees.
Note that even embedding degrees have the advantage of completely ignoring the vertical line.
However, with odd embedding degrees, we need to compute this line (which it is not vertical in the Hessian model).
%Hence, it is challenging to have the cost of odd embedding degrees being competitive with those of even embedding degrees.

Table~\ref{tbl-cmp} shows cost of pairing computation from previous work and this work.
The first column categorizes embedding degrees $k$.
The second column lists different curve models where 
$\mathcal{J}$, $\mathcal{P}$, $\mathcal{E}$, $\mathcal{H}$ denote
Weierstrass model with Jacobian coordinates,
Weierstrass model with projective coordinates,
Edwards model and
Hessian model respectively.
Note that this work considers the generalization of Hessian which is ``twisted'' Hessian curves,
and focuses on twisted $d = 3$.
The third and fourth columns shows the cost of doubling (DBL) and mixed addition (mADD) including the computing of line functions for pairing computations.
Mixed addition means $z$ coordinate equals 1.
We denote the cost of field multiplication and field squaring over the base field $\F{p}$ by $\mul$ and $\sqr$ respectively.
The cost of multiplication by curve parameters is omitted from the table.

\begin{table}[h]
\centering
\caption{Cost of pairing computation}

\begin{tabular}{ l | l | l | l}
\hline
\multicolumn{1}{c|}{$k$}
&\multicolumn{1}{c|}{Curve models}	&\multicolumn{1}{c|}{DBL}	&\multicolumn{1}{c}{mADD}	\\
\hline
\multicolumn{1}{c|}{\multirow{6}{*}{even}}
&$\mathcal{J}$, \cite{2008/IonicaJoux08} \cite{2009/fastertate}
%				&$k\mul + 1\mul + 11\sqr + 1\mulp{a}$	&$k\mul + 6\mul + 6\sqr$	\\
				&$k\mul + 1\mul + 11\sqr	$	&$k\mul + 6\mul + 6\sqr$	\\
&$\mathcal{J},a = -3$, \cite{2009/fastertate}
				&$k\mul + 6\mul + 5\sqr$		&$k\mul + 6\mul + 6\sqr$	\\
&$\mathcal{J},a = 0$, \cite{2009/fastertate}		
				&$k\mul + 3\mul + 8\sqr$		&$k\mul + 6\mul + 6\sqr$	\\
&$\mathcal{P},a = 0, b = c^2$, \cite{2009/craig}
%				&$k\mul + 3\mul + 5\sqr$		&$k\mul + 10\mul + 2\sqr + 1\mulp{b}$	\\
				&$k\mul + 3\mul + 5\sqr$		&$k\mul + 10\mul + 2\sqr$	\\
&$\mathcal{E}$, \cite{2009/fastertate}			
				&$k\mul + 6\mul + 5\sqr$		&$k\mul + 12\mul$	\\
&$\mathcal{H}$, \cite{2010/Gu}	&$k\mul + 3\mul + 6\sqr$		&$k\mul + 10\mul$	\\
\hline
%\multirow{2}{*}{\rotatebox{90}{$k{=}3$}}
\multicolumn{1}{c|}{\multirow{2}{*}{odd}}
%&$\mathcal{P}$, \cite{2010/CLN}	&$k\mul + 6\mul + 7\sqr + 1\mul_b$	&$k\mul + 13\mul + 3\sqr$	\\
&$\mathcal{P}$, \cite{2010/CLN}	&$k\mul + 6\mul + 7\sqr$		&$k\mul + 13\mul + 3\sqr$	\\
&$\mathcal{H},d=3$, this paper
%				&$k(\frac{14}{3}\mul + \frac{1}{3}\sqr) + 5\mul + 3\sqr + 1\mul_a$
				&$k(\frac{14}{3}\mul {+} \frac{1}{3}\sqr) {+} 5\mul {+} 3\sqr$
%									&$k(\frac{14}{3}\mul + \frac{1}{3}\sqr) + 11\mul + 1\mul_a$	\\
									&$k(\frac{14}{3}\mul {+} \frac{1}{3}\sqr) {+} 11\mul$	\\
\hline
\end{tabular}
\label{tbl-cmp}

%\justify
%Note: ``Pre cost'' denotes the precomputation cost.
%This cost is already included in the first column.

\end{table}

Let $\Mul$ and $\Sqr$ denote the cost of field multiplication and field squaring over the extension field.
In the doubling step (line 2 in Algorithm~\ref{algo:miller},
it has to compute $f^2 \cdot l$.
Therefore, the extra cost of $1M + 1S$ is required.
Similarly, in the addition step (line 5 in Algorithm~\ref{algo:miller},
it has to compute $f \cdot l$.
Therefore, the extra cost of $1M$ is required.
These extra costs are the same for all curve models regardless of embedding degrees.
Hence, we omit these costs from the table.

Note that there are various advantages of our constructions that cannot be justified by the cost shown in Table~\ref{tbl-cmp}.
First of all, the costs shown in Table~\ref{tbl-cmp} do not reflect the total cost for the pairing-based protocols.
This is because there are usually a few pairing computation while a lot of group operations prior to pairings.
Second, twisted Hessian curves allow faster point arithmetic operations than Weierstrass curves.
Thus, those group operations prior to pairing can take advantage of fast arithmetic on twisted Hessian curves.
Third, having curves represent in the same models for $\mathbb{G}_1$ and $\mathbb{G}_2$ does not acquire the extra cost of conversion between curve models.
As proven in~\cite{2013/bos-pairing} that either $\mathbb{G}_1$ or $\mathbb{G}_2$ of BN, BLS, and KSS curves have to be represented in Weierstrass form.
If these curve families want to take advantage of faster arithmetic on other curve shapes, this incurs an extra cost of curve shape conversion.


