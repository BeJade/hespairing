\section{Comparison}
\label{sec:cmp}

As this paper primarily concerns cubic twists, 
we only discuss results for embedding degrees that are divisible by 3.
To our knowledge, most of the previous work
on the optimization of operation counts for one iteration of Miller's loop
concentrated on pairings for $\G{1} \times \G{2}$. 
%Of course
To properly compare different results, we need to take
into account the number of iterations of Miller's loop, which differs
greatly between $\G{1} \times \G{2}$ and $\G{2} \times \G{1}$.

For pairings on $\G{1} \times \G{2}$, the lowest number of iterations occurs for the twisted ate pairing when twists are available, or the reduced Tate pairing when twists are not available.
In this paper, we explicitly address the first case, so the twisted ate pairing gives the minimal number of iterations. 
Let $t$ be the trace of Frobenius, let $T = t-1$, and let $d$ be the degree of the twist. The number of iterations of Miller's loop for the twisted ate pairing is given by $\log(T_e)$, where
$T_e \equiv T^e \pmod{r}$ and $1 < e | d$. 
Also $T$ is a $d$-th root of unity in $\F{r}$, so when $d = 6$ the smallest value of
$\log(T_e)$ is $\log(T_2) \approx \log(r)/3$, and when $d = 3$ the smallest value of $\log(T_e)$ is
$\log(T_3) \approx \log(r)$. For more details on the twisted ate pairing see~\cite{2006/hess}.

For pairings on $\G{2} \times \G{1}$, the lowest number of iterations occurs for the optimal ate pairing. The best-case-scenario (which can in principle occur for any embedding degree) is $\log(r)/\varphi(k)$ iterations of Miller's loop, where $\varphi$ is the Euler $\varphi$-function. This scenario takes $x$ as the input for the Miller's algorithm (e.g., in place of $r = r(x)$ as in Tate). For more details on the optimal ate pairing see~\cite{2010/vercauteren}. 

We compared previous results in this area for 
Weierstrass curves with Jacobian coordinates~\cite{2008/IonicaJoux08}~\cite{2009/fastertate},
Weierstrass curves with projective coordinates~\cite{2009/craig},
Edwards curves~\cite{2009/fastertate},
Edwards curves with sextic twists~\cite{2014/LWZ},
and Hessian curves with quadratic twists~\cite{2010/Gu}.
Most of these papers considered only pairings on $\G{1} \times \G{2}$ 
(many of them were written before Vercauteren's paper~\cite{2010/vercauteren} on optimal pairings)
and only even embedding degree (to avoid dealing with denominators).

\subsection{Comparing results for $\G{2} \times \G{1}$}

The only other paper containing operation counts for pairings on $\G{2} \times \G{1}$
and embedding degree divisible by 3, to our knowledge, is~\cite{2009/craig}, which considers projective Weierstrass coordinates.
In that paper they look at even embedding degrees, so we only compare our results
for the optimal ate pairing when $k=12$ and 24 (c.f. Construction 3).
Assume for simplicity that $\sqr \approx 0.8\mul$.
The formulas presented in~\cite{2009/craig} give an operation count of
$$\tfrac{41}{36}\Mul + \tfrac{41}{36}\Sqr \approx
\left\{ \begin{array}{cc}
295.2\mul & k=12 \\
1180.8\mul & k=24
\end{array} \right. 
\text{ for Miller doubling and}$$ 
$$\tfrac{4}{3}\Mul + \tfrac{1}{18}\Sqr \approx
\left\{ \begin{array}{cc}
198.4\mul & k=12 \\
793.6\mul & k=24
\end{array} \right. 
\text{ for Miller addition}.$$ 
The formulas presented in this paper give an operation count of
$$\tfrac{k}{2}\mul + \Mul + \tfrac{5}{4}\Sqr \approx
\left\{ \begin{array}{cc}
294.0\mul & k=12 \\
1164.0\mul & k=24
\end{array} \right. 
\text{ for Miller doubling and}$$ 
$$\tfrac{2k}{3}\mul + \tfrac{16}{9}\Mul \approx
\left\{ \begin{array}{cc}
264.0\mul & k=12 \\
1040.0\mul & k=24
\end{array} \right. 
\text{ for Miller addition.}$$ 
As the formulas for Hessian form are faster for doubling but slower for adding (with respect to projective Weierstrass form), there is a trade-off to assess.
Suppose that we wish to compute the optimal ate pairing and that we have 
an example for which the input for Miller's algorithm is $x$. 
The pairing can then be computed in $\log(x) = \log(r)/\varphi(k)$
iterations of Miller's loop --- this amounts to $O(\log(x))$ Miller doubling steps,
$O(\mathsf{Ham}(x))$ Miller addition steps, where $\mathsf{Ham}(x)$ denotes the Hamming weight of $x$, and the final exponentiation.
%
When $k=12$, the formulas presented in~\cite{2009/craig} compute the pairing in
$\approx 295.2 \cdot O(\log(x)) + 198.4 \cdot O(\mathsf{Ham}(x))$
multiplications in $\F{q}$ and an exponentiation,
and the formulas presented in this paper compute the pairing in
$\approx 294.0 \cdot O(\log(x)) + 264.0 \cdot O(\mathsf{Ham}(x))$
multiplications in $\F{q}$ and an exponentiation.
That is, the formulas using Hessian curves outperform the projective Weierstrass curves 
only for an $x$-value such that $\log(x) > 54.67 \cdot \mathsf{Ham}(x)$.
%%This is a condition that is not hard to achieve (and is desirable for either implementation) so we assume that it is satisfied for the comparison in Table \ref{tbl-cmp1}.
%
When $k = 24$, 
%the formulas presented in~\cite{2009/craig} compute the pairing in
%$$\approx 1180.8 \cdot O(\log(x)) + 793.6 \cdot O(\mathsf{Ham}(x))$$
%multiplications in $\F{q}$ and an exponentiation,
%and the formulas presented in this paper compute the pairing in
%$$\approx 1164.0 \cdot O(\log(x)) + 1040.0 \cdot O(\mathsf{Ham}(x))$$
%multiplications in $\F{q}$ and an exponentiation.
the formulas using Hessian curves outperform the projective Weierstrass curves for an $x$-value such that $\log(x) > 14.67 \cdot \mathsf{Ham}(x)$.
%%Again, this condition is not hard to achieve (and is desirable for either implementation) so we assume that it is satisfied for the comparison in Table \ref{tbl-cmp1}.

\subsection{Comparing results for $\G{1} \times \G{2}$}

Comparing the aforementioned papers~\cite{2008/IonicaJoux08},~\cite{2009/fastertate},~\cite{2009/craig},~\cite{2014/LWZ},~\cite{2010/Gu},
and our results, we see that the fastest curve model for embedding degree divisible by 6 together with a $\G{1} \times \G{2}$ pairing
is the Edwards form with sextic twists~\cite{2014/LWZ}
using
$\left(\frac{4k}{3} + 4\right)\mul + 7\sqr + \frac{1}{3}\Mul + \Sqr$
for one Miller doubling step and
$\left(\frac{4k}{3} {+} 12\right)\mul {+} \frac{1}{3}\Mul$
for one Miller addition step.
The fastest curve model for odd embedding degree divisible by 3 together with a $\G{1} \times \G{2}$ pairing
is the projective Weierstrass form~\cite{2009/craig}
using
$(k+6)\mul + 7\sqr + \Mul + \Sqr$
for one Miller doubling step and
$(k+13)\mul + 3\sqr + \Mul$
for one Miller addition step.


\subsection{Comparing $\G{1} \times \G{2}$ and $\G{2} \times \G{1}$}

In the following table we compare the operation counts from the \emph{most efficient} 
curve shape for each subcase
(optimal ate vs. twisted ate and even vs. odd)
in what we hope is a meaningful way:
we give the number of $\F{q}$-multiplication per Miller doubling/addition multiplied by 
$\frac{1}{\log(r)} \times $ the number of iterations. We call these numbers DBLc 
(for doubling compare) and ADDc (for addition compare).
We assume here that $\sqr = 0.8\mul$ for simplicity.
%Recall that the most efficient option for each subcase was given in the previous subsections.
The most efficient option for each subcase is as follows.

\begin{table}[h]
\centering
\caption{Best operation counts for DBLc and ADDc for each embedding degree and type of pairing}

\begin{tabular}{|l| l| l| l| r| r|}
\hline
\multicolumn{1}{|c|}{$k$}
&\multicolumn{1}{|c|}{pairing} 
&\multicolumn{1}{|c|}{Model}
&\multicolumn{1}{|c|}{\# iterations}
&\multicolumn{1}{|c|}{DBLc}	
&\multicolumn{1}{|c|}{ADDc}	
\\
\hline
\multirow{3}{*}{12} & twisted ate & Edwards~\cite{2014/LWZ} & $\log(r)/3$ & 62.9 & 25.3 \\
 & optimal ate & Projective~\cite{2009/craig} & $\log(r)/4$ & 73.8 & 49.6 \\
 & optimal ate & Hessian (this paper) & $\log(r)/4$ & 73.5 & 66.0 \\
\hline
\multirow{2}{*}{15} & twisted ate & Projective~\cite{2009/craig} & $\log(r)$ & 431.6 & 255.4 \\
 & optimal ate & Hessian (this paper) & $\log(r)/8$ & 103.1 & 120.0 \\
\hline
\multirow{2}{*}{21} & twisted ate & Projective~\cite{2009/craig} & $\log(r)$ & 826.4 & 477.4 \\
 & optimal ate & Hessian (this paper) & $\log(r)/12$ & 133.8 & 155.9 \\
\hline
\multirow{3}{*}{24} & twisted ate & Edwards~\cite{2014/LWZ} & $\log(r)/3$ & 231.5 & 78.7 \\
 & optimal ate & Projective~\cite{2009/craig} & $\log(r)/8$ & 147.5 & 99.2 \\
 & optimal ate & Hessian (this paper) & $\log(r)/8$ & 140.7 & 134.0 \\ \hline
%$\begin{array}{c} \text{even,} \\ 6|k \end{array}$}}
%& $\mathcal{P},j=0$, \cite{2009/fastertate}
%& $\frac{41}{36}\Mul + \frac{41}{36}\Sqr$
%& $\frac{4}{3}\Mul + \frac{1}{18}\Sqr$ \\
%& $\mathcal{H},j=0$, this paper
%& $\frac{k}{2}\mul + \Mul + \frac{7}{6}\Sqr$
%& $\frac{2k}{3}\mul + \frac{11}{6}\Mul$ \\
%\hline
%\multicolumn{1}{|c|}{\multirow{2}{*}{
%$\begin{array}{c} \text{odd,} \\ 3|k \end{array}$}}
%%& $\mathcal{P},j=0$, \cite{2009/fastertate}
%%& $3k\mul + \frac{5}{3}\Mul + \frac{16}{9}\Sqr$
%%& $3k\mul + \frac{22}{9}\Mul + \frac{5}{9}\Sqr$ \\
%& $\mathcal{H},j=0$, this paper
%& $\frac{4k}{3}\mul + \frac{7}{3}\Mul + \frac{14}{9}\Sqr $
%& $\frac{4k}{3}\mul + 3\Mul + \frac{2}{9}\Sqr$ \\
%&&& \\
%\hline
\end{tabular}
\label{tbl-cmp1}
\end{table}

For embedding degree 12,~\cite{2014/LWZ} is clearly the most efficient.
For embedding degrees 15 and 21, our results are clearly the most efficient.
For embedding degree 24, doubling is more efficient in Hessian form with optimal ate
while addition is more efficient in Edwards form with twisted ate. 
We could assess this trade-off in a similar way to the trade-off that was required to compare results for even embedding degrees for optimal ate pairings; 
our results will outperform those of~\cite{2014/LWZ} when the Hamming weight of 
$x$ is sufficiently low compared to $\log(x)$.

Not included in Table~\ref{tbl-cmp1} are the 
precomputation costs (which are relatively low for our constructions) and the final exponentiation costs (which are roughly uniform across all curve shapes).
%and group operations on twisted Hessian curves allow faster point arithmetic operations than Weierstrass curves. -- this *is* accounted for in the table!
A significant part of the precomputation cost for many models is the conversion
between curve models, which is not necessary for our constructions.
(Recall that for BN, BLS, and KSS, this conversion is always necessary if one wants to take advantage of
the fast point arithmetic on Hessian or Edwards curves, as proven in~\cite{2013/bos-pairing}.)

%\begin{remark}
%Also not included in table~\ref{tbl-cmp1}, or indeed in this paper, is a meaningful comparison between different embedding degrees. 
%Such a comparison should take account the required base field and extension field size for each case, 
%and timing comparisons the most efficient field arithmetic for both the base field and the extension field in each case. 
%This is unfortunately beyond the scope of this paper so we leave this for future work.
%\end{remark}

%\begin{table}[h]
%\centering
%\caption{Comparison of DBL for $\G{1} \times \G{2}$ with different curve models and embedding degrees}
%
%\begin{tabular}{| l | l | l | l}
%\hline
%\multicolumn{1}{|c|}{$k$}
%&\multicolumn{1}{|c|}{Curve models}	&\multicolumn{1}{c|}{DBL}		\\
%\hline
%\multicolumn{1}{|c|}{\multirow{6}{*}{
%$\begin{array}{c} \text{even,} \\ 6|k \end{array}$}}
%&$\mathcal{J}$, \cite{2008/IonicaJoux08} \cite{2009/fastertate}
%				&$(k + 1)\mul + 11\sqr + \Mul + \Sqr$		\\
%&$\mathcal{J},a_4 = -3$, \cite{2009/fastertate}
%				&$(k + 6)\mul + 5\sqr + \Mul + \Sqr$	\\
%&$\mathcal{J},a_4 = 0$, \cite{2009/fastertate}		
%				&$(k + 3)\mul + 8\sqr + \Mul + \Sqr$		\\
%&$\mathcal{P},(a_4,a_6) = (0, b^2)$, \cite{2009/craig}
%				&$(k + 3)\mul + 5\sqr + \Mul + \Sqr$		\\
%&$\mathcal{E}$, \cite{2009/fastertate}			
%				&$(k + 6)\mul + 5\sqr + \Mul + \Sqr$			\\
%&$\mathcal{E},j=0$, \cite{2014/LWZ}
%				&$\left(\frac{4k}{3}+ 4\right)\mul + 7\sqr  + \frac{1}{3}\Mul + \Sqr$\\
%&$\mathcal{H}$, \cite{2010/Gu}	&$(k + 3)\mul + 6\sqr + \Mul + \Sqr$ 	\\
%&$\mathcal{H},j=0$, this paper 	&
%$ \left(\frac{2k}{3} + 5\right)\mul + 3\sqr + \frac{2}{3}\Mul + \Sqr$			\\
%\hline
%\multicolumn{1}{|c|}{\multirow{2}{*}{
%$\begin{array}{c} \text{odd,} \\ 3|k \end{array}$}}
%&$\mathcal{P}$, \cite{2010/CLN}	&$(k + 6)\mul + 7\sqr + \Mul + \Sqr$		\\
%&$\mathcal{H},j=0$, this paper
%&$ \left(\frac{4k}{3} + 5\right)\mul + 3\sqr + \frac{16}{9}\Mul + \frac{11}{9}\Sqr$
%\\
%\hline
%\end{tabular}
%\label{tbl-cmp1}

%\justify
%Note: ``Pre cost'' denotes the precomputation cost.
%This cost is already included in the first column.

%\end{table}

%\begin{table}[h]
%\centering
%\caption{Comparison of mADD for $\G{1} \times \G{2}$ with different curve models and embedding degrees}
%
%\begin{tabular}{| l | l | l | l}
%\hline
%\multicolumn{1}{|c|}{$k$}
%&\multicolumn{1}{|c|}{Curve models}	&\multicolumn{1}{c|}{mADD}	\\
%\hline
%\multicolumn{1}{|c|}{\multirow{6}{*}{
%$\begin{array}{c} \text{even,} \\ 6|k \end{array}$}}
%&$\mathcal{J}$, \cite{2008/IonicaJoux08} \cite{2009/fastertate}
%		&$(k+ 6)\mul + 6\sqr + \Mul $	\\
%&$\mathcal{J},a_4 = -3$, \cite{2009/fastertate}
%	&$(k+ 6)\mul + 6\sqr + \Mul$	\\
%&$\mathcal{J},a_4 = 0$, \cite{2009/fastertate}		
%	&$(k + 6)\mul + 6\sqr + \Mul$	\\
%&$\mathcal{P},(a_4,a_6) = (0, b^2)$, \cite{2009/craig}
%		&$(k + 10)\mul + 2\sqr + \Mul$	\\
%&$\mathcal{E}$, \cite{2009/fastertate}			
%		&$(k+ 12)\mul + \Mul$	\\
%&$\mathcal{E},j=0$, \cite{2014/LWZ}
%				&$\left(\frac{4k}{3} + 12\right)\mul  + \frac{1}{3}\Mul$ \\
%&$\mathcal{H}$, \cite{2010/Gu}	&$(k + 10)\mul + \Mul$	\\
%&$\mathcal{H},j=0$, this paper 	&
%$ \left(\frac{2k}{3} + 11\right)\mul + \frac{2}{3}\Mul$	\\
%\hline
%\multicolumn{1}{|c|}{\multirow{2}{*}{
%$\begin{array}{c} \text{odd,} \\ 3|k \end{array}$}}
%&$\mathcal{P}$, \cite{2010/CLN}		&$(k + 13)\mul + 3\sqr + \Mul$	\\
%&$\mathcal{H},j=0$, this paper
%&$\left(\frac{4k}{3} + 11\right)\mul + \frac{16}{9}\Mul + \frac{2}{9}\Sqr $	\\
%\hline
%\end{tabular}
%\label{tbl-cmp1a}
%
%%\justify
%%Note: ``Pre cost'' denotes the precomputation cost.
%%This cost is already included in the first column.
%
%\end{table}

%\begin{table}[h]
%\centering
%\caption{Comparison of the costs of DBL and mADD for $\G{2} \times \G{1}$ with different curve models and embedding degrees}
%
%\begin{tabular}{| l | l | l | l |}
%\hline
%\multicolumn{1}{|c|}{$k$}
%&\multicolumn{1}{|c|}{Curve models}	&\multicolumn{1}{|c|}{DBL}	&\multicolumn{1}{|c|}{mADD}	
%\\
%\hline
%\multicolumn{1}{|c|}{\multirow{2}{*}{
%$\begin{array}{c} \text{even,} \\ 6|k \end{array}$}}
%& $\mathcal{P},j=0$, \cite{2009/fastertate}
%& $\frac{41}{36}\Mul + \frac{41}{36}\Sqr$
%& $\frac{4}{3}\Mul + \frac{1}{18}\Sqr$ \\
%& $\mathcal{H},j=0$, this paper
%& $\frac{k}{2}\mul + \Mul + \frac{7}{6}\Sqr$
%& $\frac{2k}{3}\mul + \frac{11}{6}\Mul$ \\
%\hline
%\multicolumn{1}{|c|}{\multirow{2}{*}{
%$\begin{array}{c} \text{odd,} \\ 3|k \end{array}$}}
%%& $\mathcal{P},j=0$, \cite{2009/fastertate}
%%& $3k\mul + \frac{5}{3}\Mul + \frac{16}{9}\Sqr$
%%& $3k\mul + \frac{22}{9}\Mul + \frac{5}{9}\Sqr$ \\
%& $\mathcal{H},j=0$, this paper
%& $\frac{4k}{3}\mul + \frac{7}{3}\Mul + \frac{14}{9}\Sqr $
%& $\frac{4k}{3}\mul + 3\Mul + \frac{2}{9}\Sqr$ \\
%&&& \\
%\hline
%\end{tabular}
%\label{tbl-cmp2}

%\justify
%Note: ``Pre cost'' denotes the precomputation cost.
%This cost is already included in the first column.

%\end{table}


%As explained in Section~\ref{subsec:w2h}, 
%this conversion costs $9\Mul + 2\Sqr + 5\mul$ plus one inversion and one cube root computation.

%\input numer

%%%%%%%%%%%%%%%%%%%%%%%%%%%%%%%%%%%%%%%%%%%%%%%%%%

%\subsection{Numerical comparison}
%Since we would like to focus on a better balance of field size and embedding degree,
%we would like to focus on odd embedding degrees with allow, for example, $k = 15, 21$.
%Therefore, we think it is the best to compare our result of odd embedding degree twist Hessian curves,
%with odd embedding degree Weierstrass curve~\cite{2010/CLN}.
%
%In many protocols based on pairings, there are usually only a few pairing computation but many operations in $\G{1}$ and less in $\G{2}$, and $\G{T}$.
%In this simple comparison analysis, we would like to focus only on exponentiation $\G{1}$ because there are more of these compared to $\G{2}$ and $\G{T}$.
%We would like to show how much we save for one exponentiation in $\G{1}$.
%
%Consider embedding degree $k = 15$.
%Let $q = 384$.
%According to~\cite{EFD} and~\cite{2013/bos-pairing}, for Weierstrass curves of a form $y^2 = x^3 + b$,
%point doubling costs $2\Mul + 5\Sqr$ and point addition costs $11\Mul + 5\Sqr$.
%For our twisted Hessian curves of a form $ax^3 + y^3 + 1 = 0$,
%point doubling costs $5\Mul + 2\Sqr$ and point additio costs $11\Mul$.
%Let consider $S = 0.8M$.
%This means that on Weierstrass curve, point doubling and point addition cost $6\Mul$ and $15\Mul$ respectively,
%while on twisted Hessian curves point doubling and point addition cost $6.6\Mul$ and $11\Mul$ respectively.
%Let consider double-and-add algorithm with window with 5.
%This means that one exponentiation costs
%$384(6\Mul) + 76(15\Mul) = 3444\Mul$ on Weierstrass curves, and
%$384(6.6\Mul) + 76(11\Mul) = 3370.4$ on twisted Hessian curves.
%One pairing on Weierstrass curves cost $15\Mul + 6\Mul + 7\Sqr = 26.6\Mul$.
%One pairing on twisted Hessian curves cost $15\Mul + 6\Mul + 7\Sqr = 26.6\Mul$.
%
%Even though pairing on odd embedding degree twisted Hessian curves might cost more than that on Weierstrass curves,
%but if we consider as a whole pairing protocols, we save more on the group operations.
%The example of numerial comarison above show this.



