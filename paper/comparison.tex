\section{Comparison}
\label{sec:cmp}

We would like to emphasize that:
\begin{itemize}
\item	Most of the previous work concerned Weierstrass curves.
\item	Most of the previous work concerned even embedding degrees.
\item	There was some previous work on Hessian curves but with even embedding degrees.
\item	There was also some previous works with odd embedding degrees but using Weierstrass curves.
\end{itemize}
Therefore, it is difficult to have an appropriate comparison between the constructions in this article and previous work, as we are the first to consider the twisted Hessian curves with odd embedding degrees, to our knowledge.
Note that even embedding degrees have the advantage of being able to completely eliminate denominators in the computation of the Tate pairing. However, with odd embedding degrees, we need to compute the denominator as it is non-trivial.
%Hence, it is challenging to have the cost of odd embedding degrees being competitive with those of even embedding degrees.

Table~\ref{tbl-cmp} shows the costs of the computation of pairings using our constructions in comparison with previous work. The first column gives the embedding degrees $k$.
The second column gives the curve model, where 
$\mathcal{J}$, $\mathcal{P}$, $\mathcal{E}$, $\mathcal{H}$ denote the
Weierstrass model with Jacobian coordinates, the
Weierstrass model with projective coordinates,
the Edwards model and the
Hessian model respectively.
Note that this work considers the generalization of Hessian curves to ``twisted'' Hessian curves, and that the twists are of degree 3.
The third and fourth columns show the cost of doubling (DBL) and mixed addition (mADD) including the computation of the line functions for pairing computations.
By `mixed addition' we mean that the $z$ coordinate is set to be 1.
We denote the cost of field multiplication and field squaring over the base field $\F{p}$ by $\mul$ and $\sqr$ respectively.
The cost of multiplication by curve parameters is omitted from the table.

\begin{table}[h]
\centering
\caption{Cost of pairing computation}

\begin{tabular}{ l | l | l | l}
\hline
\multicolumn{1}{c|}{$k$}
&\multicolumn{1}{c|}{Curve models}	&\multicolumn{1}{c|}{DBL}	&\multicolumn{1}{c}{mADD}	\\
\hline
\multicolumn{1}{c|}{\multirow{6}{*}{even}}
&$\mathcal{J}$, \cite{2008/IonicaJoux08} \cite{2009/fastertate}
%				&$k\mul + 1\mul + 11\sqr + 1\mulp{a}$	&$k\mul + 6\mul + 6\sqr$	\\
				&$k\mul + 1\mul + 11\sqr	$	&$k\mul + 6\mul + 6\sqr$	\\
&$\mathcal{J},a = -3$, \cite{2009/fastertate}
				&$k\mul + 6\mul + 5\sqr$		&$k\mul + 6\mul + 6\sqr$	\\
&$\mathcal{J},a = 0$, \cite{2009/fastertate}		
				&$k\mul + 3\mul + 8\sqr$		&$k\mul + 6\mul + 6\sqr$	\\
&$\mathcal{P},a = 0, b = c^2$, \cite{2009/craig}
%				&$k\mul + 3\mul + 5\sqr$		&$k\mul + 10\mul + 2\sqr + 1\mulp{b}$	\\
				&$k\mul + 3\mul + 5\sqr$		&$k\mul + 10\mul + 2\sqr$	\\
&$\mathcal{E}$, \cite{2009/fastertate}			
				&$k\mul + 6\mul + 5\sqr$		&$k\mul + 12\mul$	\\
&$\mathcal{H}$, \cite{2010/Gu}	&$k\mul + 3\mul + 6\sqr$		&$k\mul + 10\mul$	\\
\hline
%\multirow{2}{*}{\rotatebox{90}{$k{=}3$}}
\multicolumn{1}{c|}{\multirow{2}{*}{odd}}
%&$\mathcal{P}$, \cite{2010/CLN}	&$k\mul + 6\mul + 7\sqr + 1\mul_b$	&$k\mul + 13\mul + 3\sqr$	\\
&$\mathcal{P}$, \cite{2010/CLN}	&$k\mul + 6\mul + 7\sqr$		&$k\mul + 13\mul + 3\sqr$	\\
&$\mathcal{H},d=3$, this paper
%				&$k(\frac{14}{3}\mul + \frac{1}{3}\sqr) + 5\mul + 3\sqr + 1\mul_a$
				&$k(\frac{14}{3}\mul {+} \frac{1}{3}\sqr) {+} 5\mul {+} 3\sqr$
%									&$k(\frac{14}{3}\mul + \frac{1}{3}\sqr) + 11\mul + 1\mul_a$	\\
									&$k(\frac{14}{3}\mul {+} \frac{1}{3}\sqr) {+} 11\mul$	\\
\hline
\end{tabular}
\label{tbl-cmp}

%\justify
%Note: ``Pre cost'' denotes the precomputation cost.
%This cost is already included in the first column.

\end{table}

Let $\Mul$ and $\Sqr$ denote the cost of field multiplication and field squaring over the extension field.
In the doubling step (line 2 in Algorithm~\ref{algo:miller}),
we have to compute $f^2 \cdot l$.
Therefore, the extra cost of $1\Mul + 1\Sqr$ is required.
Similarly, in the addition step (line 5 in Algorithm~\ref{algo:miller}),
we have to compute $f \cdot l$.
Therefore, the extra cost of $1\Mul$ is required.
These extra costs are the same for all curve models regardless of the embedding degree, hence, we omit these costs from the table.

Note that there are various advantages of our constructions that cannot be justified by the cost shown in Table~\ref{tbl-cmp}.
First of all, the costs shown in Table~\ref{tbl-cmp} do not reflect the total cost for the pairing-based protocols.
This is because generically there are many group operations performed prior to relatively few pairing computations, and group operations on twisted Hessian curves allow faster point arithmetic operations than Weierstrass curves.
Secondly, having curves represented in the same models for $\mathbb{G}_1$ and $\mathbb{G}_2$ does not induce the extra cost of conversion between curve models. (Recall that for BN, BLS, and KSS, this conversion is always necessary if one wants to take advantage of the fast point-arithmetic on Hessian or Edwards curves, as proven in~\cite{2013/bos-pairing}.)


