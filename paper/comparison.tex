\section{Comparison}

As previously described in Section~\ref{sec:intro} that:
\begin{itemize}
\item	most of the previous works concerned Weierstrass curves
\item	most of the previous works concerned even embedding degrees
\item	there were some previous works on Hessian curves but with even embedding degrees
\item	there were also some previous works with odd embedding degrees but usisng Weierstrass curves
\end{itemize}
it is difficult to have a fair comparison between this work and previous works
since this work happens to be the first to consider the twisted Hessian curves with odd embedding degrees.
Note that even embedding degrees have the advantage of completely ignoring the vertical line.
But in odd embedding degrees, we need to compute that line (note that it is not a vertical line in the Hessian model).
Therefore, the cost is higher in odd than in even embedding degrees.

Table~\ref{tbl-cmp} shows costs of pairing computation of 
``DBL'' means doubling
``mADD'' means mixed addition ($z$ coordinate is 1)
$\mathcal{J}$ means Weierstrass model with Jacobian coordinates
$\mathcal{P}$ means Weierstrass model with projective coordinates
$\mathcal{E}$ means Edwards model
$\mathcal{H}$ means Hessian model
$k$ stands for embedding degree
$d$ is the twist degree
\mul and \sqr is the cost of field multiplication and field squaring over the base field $\F{q}$

$\Mul$ and $\Sqr$ denote the cost of field multiplication and field squaring over the extension field.
Doubling has to compute $f^2 \cdot l$.
Therefore, the extra cost of $1M + 1S$ is required.
Addition has to compute $f \cdot l$.
Therefore, the extra cost of $1M$ is required.
These extra costs are the same for all curve models regardless of embedding degrees.
Therefore, we omit these costs from the table.

\begin{table}[h]
\centering
\caption{Cost of pairing computation}

\begin{tabular}{ l | l | l | l}
\hline
\multicolumn{1}{c|}{$k$}
&\multicolumn{1}{c|}{Curve models}	&\multicolumn{1}{c|}{DBL}	&\multicolumn{1}{c}{mADD}	\\
\hline
\multicolumn{1}{c|}{\multirow{6}{*}{even}}
&$\mathcal{J}$, \cite{2008/IonicaJoux08} \cite{2009/fastertate}
				&$k\mul + 1\mul + 11\sqr + 1\mulp{a}$	&$k\mul + 6\mul + 6\sqr$	\\
&$\mathcal{J},a = -3$, \cite{2009/fastertate}
				&$k\mul + 6\mul + 5\sqr$		&$k\mul + 6\mul + 6\sqr$	\\
&$\mathcal{J},a = 0$, \cite{2009/fastertate}		
				&$k\mul + 3\mul + 8\sqr$		&$k\mul + 6\mul + 6\sqr$	\\
&$\mathcal{P},a = 0, b = b^2$, \cite{2009/craig}
				&$k\mul + 3\mul + 5\sqr$		&$k\mul + 10\mul + 2\sqr + 1\mulp{b}$	\\
&$\mathcal{E}$, \cite{2009/fastertate}			
				&$k\mul + 6\mul + 5\sqr$		&$k\mul + 12\mul$	\\
&$\mathcal{H}$, \cite{2010/Gu}	&$k\mul + 3\mul + 6\sqr$		&$k\mul + 10\mul$	\\
\hline
%\multirow{2}{*}{\rotatebox{90}{$k{=}3$}}
\multicolumn{1}{c|}{\multirow{2}{*}{odd,$d{=}3$}}
&$\mathcal{P}$, \cite{2010/CLN}	&$k\mul + 6\mul + 7\sqr + 1\mul_b$	&$k\mul + 13\mul + 3\sqr$	\\
&$\mathcal{H}$, this paper
				&$k(2\mul + \frac{2}{3}\sqr) + 5\mul + 3\sqr + 1\mul_a$
									&$k(2\mul + \frac{2}{3}\sqr) + 11\mul + 1\mul_a$	\\
%				&$1\Mul + 1\Sqr + \frac{5}{3}k\mul + \frac{2}{3}k\sqr + 5\mul + 3\sqr$
%											&$1\Mul + \frac{5}{3}k\mul + \frac{2}{3}k\sqr + 11\mul$	\\
\hline
\end{tabular}
\label{tbl-cmp}

%\justify
%Note: ``Pre cost'' denotes the precomputation cost.
%This cost is already included in the first column.

\end{table}


