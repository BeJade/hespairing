\section{Computation of line functions}
\label{sec:lines}

The efficiency of pairings substantially relies on the computation of the line functions
(as denoted by $\ell_{2R} = l_{2R}/v_{2R}$ and $\ell_{R,P} = l_{R,P}/v_{R,P}$ in Algorithm~\ref{algo:miller}).
Recall that an equation of a line is simply $y = mx + b$.
Therefore, constructing the line $l$ passing through points $P$ and $R$
is done by substituting the coordiinates $x$ and $y$ of points $P$ and $R$
then solving for $m$ and $b$.
Once the line is constructed, it is then evaluated at another point $Q$

Specifically, to compute the line $l_{R,P}$ passing through
points $R = (x_R, y_R)$ and $P = (x_P, y_P)$ where $R \ne P$
and evaluated at point $Q = (x_Q, y_Q)$,
we perform the following computations.
%let $P = (x_P, y_P)$ and  $R = (x_R, y_R)$ where $P \ne R$.
%Let $Q = (x_Q, y_Q)$ be the point to evaluate.
%The line $l_{R,P}$ evaluated at $Q$ is constructed as follows.
\begin{align*}
m_{l_{R,P}} &= (y_P - y_R)/(x_P - x_R)			\\
b_{l_{R,P}} &= y_P - m_{l_{R,P}} \cdot x_P		\\
l_{R,P} &= y_Q - (m_{l_{R,P}} \cdot x_Q + b_{l_{R,P}}) 
\end{align*}
Note that %for this case where $P \ne R$,
the line passing thorugh points $P$ and $R$ is also passing through point $-(P+R)$.
Using any two out of these three points would result in the same line $l$.

To compute the line $l_{2R}$ tangent at point $R$,
we use the fact that this line also intersects the curve at point $-(2R)$.
Therefore, we construct a line passing through points $R$ and $-(2R)$.
Recall that for curves in twisted Hessian forms,
a negation of a point $2R = (x_{2R}, y_{2R})$ is $-2R = (x_{2R}/y_{2R}, 1/y_{2R})$.
Let $R = (x_{R}, y_{R})$
and $-2R = (x_{2R}/y_{2R}, 1/y_{2R})$. % = (x'_{2R},y'_{2R})$.
The line $l$ tangent at point $R$ evaluated at point $Q$ is computed as follows.
\begin{align*}
m_{l_{2R}} &= \bigg( \frac{1}{y_{2R}} - x_{R} \bigg) / \bigg( \frac{x_{2R}}{y_{2R}} - x_{R} \bigg)	\\
b_{l_{2R}} &= y_R - m_{l_{2R}} \cdot x_R		\\
l_{2R} &= y_Q - (m_{l_{2R}} \cdot x_Q + b_{l_{2R}})
\end{align*}

To construct the line $v$ passing through points $(P+R)$ and its negation $-(P+R)$,
recall that for Weierstrass curves,
this line $v$ is simply a vertical line
and thus of the form $y = c$.
However, for the (twisted) Hessian curves, the line $v$ is not a vertical line
and thus has to be constructed in a similar way as the line $l$,
i.e., solving for $m$ and $b$ of $y = mx + b$ using points $P+R$ and $-(P+R)$.
Note also that the line passing through the point and its negative also passes through
the point $(0,-1)$, the neutral point of twisted Hessian forms.
% (0 : -1 : 1)

Let $(P+R) = (x_{RP},y_{RP})$.
By using the neutral point $(0,-1)$
the line $v_{R,P}$ is computed as follows.
\begin{align*}
m_{v_{R,P}} &= (y_{RP} + 1) / x_{RP}	\\
b_{v_{R,P}} &= -1			\\
v_{R,P} &= y_Q - (m_{v_{R,P}} \cdot x_Q + b_{v_{R,P}})
\end{align*}
The advantage of using the neutral point is that
it saves some computations and $b_{v_{R,P}}$ becomes $-1$
due to a multiplication by zero.

For the line $v_{2R}$,
the calculation is similar to $v_{R,P}$.
Simply replace $(P+R)$ by $(2R) = (x_{2R}, y_{2R})$,
and compute as above.

Note that the presented calculations,
we use points represented in affine coordinates,
e.g., $P = (x,y)$.
To avoid inversion, e,g, when computing $m$,
we use projective coordinates where
$x = X/Z$ and $y = Y/Z$,
and thus points are represented as, e.g., $P = (X : Y : Z)$.
The formulas given in the following section use projective coordinates.

Recall that the lines $v_{R,P}$ and $v_{2R}$ appear as denorminators,
namely, we have to compute $l_{2R}/v_{2R}$ and $l_{R,P}/v_{R,P}$.
%Because computing inversion is expensive,
%we apply ``denominotr elimination'' to optimize this inversion.
Note that for even embedding degree,
this line $v$ lies in a subfield of $\F{q^k}$
which will be $1$ after computing the final exponentiation
(line 8 in Algorithm~\ref{algo:miller} in Section~\ref{sec:background}).
Therefore, for even embedding degrees, the line $v$ can be omitted.
However, for odd embedding degrees, we cannot neglect the denorminator $v$.
Because inversioin is expensive,
we first apply ``denominator ellimination'' technique to optimize this inversion.


%%XXX To Chloe: is this still correct? wouldn't everything collapse to 1?
%%XXX don't forget to put reference
%Let notation be as above, and let $P \in \mathbb{G}_{1}$. We now move to the explicit computation of the line functions
%$\ell_{1} = L_{iP,jP}$ and $\ell_{2} = L_{(i+j)P,-(i+j)P}$
%given in the formula for the reduced Tate pairing;
%recall from Section \ref{sec:Tate} that computing these line functions is enough to compute the reduced Tate pairing.
%Recall also that the line $\ell_{2}$ is in the denominator of the pairing. Inversion is expensive,
%so we first apply `denominator elimination' to optimize this inversion.
%
%%%%%%%%%%%%%%%%%%%%%
%
%%Let $m = (m_{\ell-1}, \dots, m_1, m_0)$ be the binary representation of $m$.
%%Initialize $R = P$ and $f = 1$.
%%Algorithm~\ref{algo:miller} shows Miller's algorithm.
%%%\renewcommand{\algorithmicrequire}{\textbf{Initialize:}}
%%%\renewcommand{\algorithmicensure}{\textbf{Output:}}
%%\begin{algorithm}
%%\caption{Miller's algorithm}
%%\label{algo:miller}
%%\begin{algorithmic}[1]
%%
%%%	\Require $R = P$ and $f = 1$
%%
%%	\For {$i := \ell - 2$ {\bf{down to}} $0$}
%%		\State $f \leftarrow f^2 \cdot f_{DBL(R)}(Q)$
%%		\State $R \leftarrow 2R$
%%		\If {$m_i = 1$}
%%			\State $f \leftarrow f \cdot f_{ADD(R,P)}(Q)$
%%			\State $R \leftarrow R+P$
%%		\EndIf
%%	\EndFor
%%	\State $f \leftarrow f^{(p^k-1)/r}$
%%
%%\end{algorithmic}
%%\end{algorithm}

%%%%%%%%%%%%%%%%%%%%%%%%%%%%%%%%%%%%%%%%%%%%%%%%%%%%%%%%%%%%%%%%%%%%%%%%%%%%%%%%%%%%%%%%%%%%%%%%%%%%

%\subsection{Denominator elimination}
We follow a similar technique as described in~\cite{2008/lin} and~\cite{2009/deg15}
by rewriting an inversion into a fraction for which the denominator lies in a subfield.
Define a field extension $\F{p^k}$ of a finite field $\F{p}$ and suppose that $x,y \in \F{p^k}$.

Observe that
$$ \frac{1}{x-y} = \frac{x^2 + xy + y^2}{x^3 - y^3}. $$
As $x^3$ and $y^3$ lie in a subfield of $\F{p^k}$, we have that
\[(x^{3} - y^{3})^{\frac{p^{k}-1}{r}} = 1,\]
hence, after the final exponentiation in the computation of the pairing, this factor does not appear.
This means that we can ignore the computation of the denominator $x^3 - y^3$
Therefore, instead of computing $\frac{1}{x-y}$, we compute $x^2 + xy + y^2$.
The cost of division by $x - y$ then becomes the cost of multiplication by $x^2 + xy + y^2$.

%%%%%%%%%%%%%%%%%%%%%%%%%%%%%%%%%%%%%%%%%%%%%%%%%%%%%%%%%%%%%%%%%%%%%%%%%%%%%%%%%%%%%%%%%%%%%%%%%%%%

