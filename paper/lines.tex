\section{Computation of line functions}
\label{sec:lines}

Each iteration of Miller's loop (Algorithm~\ref{algo:miller}) includes a 
\emph{Miller doubling} step and some of the iterations also include a
\emph{Miller addition} step. 
The Miller doubling step has four costly parts:
computing the double of a point $R$ on the curve, computing the 
Miller function
$\ell_{R,R} = l_{R,R}/v_{2R}$, squaring an element $f \in \F{q^k}$, 
and multiplying $f^2$ by $\ell_{R,R}$. 
The Miller addition step has three costly parts:
computing the sum of two points $P$ and $R$ on the curve, 
computing the Miller function
$\ell_{P,R} = l_{P,R}/v_{P+R}$,
and multiplying an element $f \in \F{q^k}$ by $\ell_{P,R}$. 
We attempt in the following sections to optimize each of these parts
for
Hessian curves 
\[\mathcal{H}/\F{q}: X^3 + Y^3 + Z^3 = 0\]
of $j$-invariant 0 
for pairings on both $\G{1} \times \G{2}$ (such as the Tate pairing)
and $\G{2} \times \G{1}$ (such as the ate pairing).

%The efficiency of pairings relies heavily on the computation of the Miller function
%(as denoted by $\ell_{2R} = l_{R,R}/v_{2R}$ and $\ell_{R,P} = l_{R,P}/v_{R+P}$ in Algorithm~\ref{algo:miller}).
%Recall that $l_{R,P}$ is notation for the line passing through $R$ and $P$ and that
%$v_{R}$ is notation for the line passing through $R$ and $-R$.
%In many other representations of elliptic curves, for example short Weierstrass form or
%Edwards form, the negation of a point is its reflection in the $x$-axis
%due to the fact that the identity group element is the point at infinity.
%In particular, the line $v_R$ has a particularly simple form, meaning that very often
%the field inversion can be avoided using demoninator elimination techniques.

\subsection{Denominator elimination}\label{denelim}

It is of course desirable to avoid the field inversion that comes from dividing by 
$v_{P_1 + P_2}(Q)$, with $P_2 = R$ and $P_1 \in \{R,P\}$, which we can do (to some extent).
For curves in (twisted) Hessian form, the neutral group element is given by 
$(0:-1:1)$,
and negation by
\[-(x,y) = (x/y,1/y)\]
(in affine coordinates). 
This means that the line $v_{P_1 + P_2}$
passing through $P_3 = P_1 + P_2$ and $(0:-1:1)$ has a more complicated form than for many other popular curve shapes (such as
short Weierstrass or Edwards).
Namely, writing 
$(X_3:Y_3:Z_3) = P_3$ and $(X_Q:Y_Q:Z_Q)$, we have
\[v_{P_3}(Q): (Z_3 + Y_3)X_Q - (Z_Q+Y_Q)X_3.\]

When considering pairings on $\G{1} \times \G{2}$, we have that $P_3 \in \G{1}$ and $Q \in \G{2}$, and when considering pairings on $\G{2} \times \G{1}$, we have that $P_3 \in \G{2}$ and $Q \in \G{1}$.
As $v_{P_3}(Q) = v_{Q}(P_3)$, exactly the same arguments apply to $\G{1} \times \G{2}$ as to $\G{2} \times \G{1}$ in this case; 
say for simplicity that $P_3 \in \G{1}$ and $Q \in \G{2}$.
Suppose that we have chosen $Q$ such that there exists $Q' \in \mathcal{H}_a(\F{q^{k/3}})$ for which $Q = \varphi(Q')$, where $\varphi$ is the 
cubic twist isomorphism from Equation \ref{twistiso}.

\paragraph*{Even embedding degrees.}

The following is essentially as rephrasing of the denominator elimination technique presented in~\cite{2010/Gu} (although they do not mention pairings on $\G{2} \times \G{1}$).

Assume now that $6|k$. In particular, by the discussion in
Section~\ref{sextic}, the triangular curve
\[T:\alpha^2VW(V + \omega^3W) = U^3,\]
with $\alpha$ and $\omega$ as in Section~\ref{sextic},
defines a quadratic twist of $\mathcal{H}_{\omega^3}$ via the isomorphism
$\psi$ of Equation~\ref{quadtwist}.
We choose our point $Q' \in \mathcal{H}_{\omega^3}(\F{q^{k/3}})$ from the
image under $\psi$ of $T(\F{q^{k/6}})$, so that there exist $U, V, W \in \F{q^{k/6}}$ for which
$$Q' = (U:\beta(\alpha V - 54W) : \beta(-\alpha V + 54\zeta_3^2 W)),$$
where $\beta = \zeta_3-\zeta_3^2$ and 
$\zeta_3 \in \F{q}$ is a primitive cube root of unity.
Evaluation of $v_{P_3}$ at $Q = \varphi(Q')$ then gives
\[v_{2R}(Q) : (Z_3 + Y_3) U\omega - 54\beta(\zeta_3^2-1)W X_3 \in \F{q^{k/2}}.\]
This value will go to 1 in the final expontentiation step of Miller's algorithm 
(Algorithm~\ref{algo:miller}), so without loss of generality we can set it to 1 throughout the computation.

\paragraph*{Odd embedding degrees.}

Unfortunately the denominator elimination technique of~\cite{2010/Gu} does not apply to this case;
instead we extend ideas of~\cite{2008/lin} and~\cite{2009/deg15}.

Observe that
\[\frac{1}{x-y} = \frac{x^2 + xy + y^2}{x^3-y^3}.\]
Let $Q' = (X_{Q'},Y_{Q'},Z_{Q'})$.
Plugging $x = (Z_3 + Y_3)X_{Q'}\omega$ and $y = (Z_{Q'}+Y_{Q'})X_3$ in $\frac{1}{v_{P_3}(Q)}$ with $Q = \varphi(Q')$,
we get that the denominator $x^3 - y^3$ is in $\F{q^{k/3}}$ so will go to 1 in the final exponentiation, hence can be set to 1 for the whole computation.
That is, we replace $\frac{1}{v_{P_3}(Q)}$ by the numerator 
$$n_{P_3}(Q) = ((Z_3 + Y_3)X_{Q'})^2\omega^2 
+ (Z_3 + Y_3)X_{Q'}(1+Y_{Q'})X_3\omega 
+ ((1+Y_{Q'})X_3)^2$$
and we replace the Miller function 
$\ell_{P_1,P_2}(Q)$ by $n_{P_3}(Q) \cdot l_{P_1,P_2}(Q)$.
The numerator $n_{P_3}(Q)$ can be computed with cost
$\frac{2k}{3}\mul + \frac{2}{9}\Sqr + \frac{1}{9}\Mul$ via
\[\begin{array}{cccc}
u = (Z_3 + Y_3)\cdot X_{Q'}; &
v = (1+Y_{Q'})X_3; &
n = u^2\omega^2 + (u\cdot v)\omega + v^2.
\end{array}\]

\subsection{Miller doubling}

Let $R = (X_1 : Y_1 : Z_1) \in \mathcal{H}_b(K)$ for $b \in \{1,a\}$.
The fastest known formulas to compute $2R = (X_3 : Y_3 : Z_3)$
(due to~\cite{2010/hess}) are as follows:

\begin{align*}
T &= Y_1^2;\	\qquad	A = Y_1 \cdot T;\	\qquad
S = Z_1 ^ 2;\	\qquad	B = Z_1 \cdot S;\\
X_3 &= X_1 \cdot (A - B);\	\quad
Y_3 = -Z_1 \cdot (2A + B);\	\quad
Z_3 = Y_1 \cdot (A + 2B).
\end{align*}
The cost for point doubling with the above formulas is 5 multiplications in $K$ and 2 squarings in $K$.

In all that follows we denote multiplication and squaring in $\F{q}$ by $\mul$ and $\sqr$ respectively, and multiplication and squaring in $\F{q^k}$ by $\Mul$ and $\Sqr$ respectively. We also assume always that $3|k$.

\subsubsection{Pairings on $\G{1} \times \G{2}$}

Recall that Miller's (doubling) function is given by
$$\ell_{R,R}(Q) = l_{R,R}(Q)/v_{2R}(Q),$$
For pairings on $\G{1} \times \G{2}$ the input points are
$P \in \G{1}$ and $Q \in \G{2}$, and
$R$ will be a multiple of $P$.

We first address the computation of $l_{R,R}(Q)$.
This line is the tangent line to $\mathcal{H}_1$ at $R$ evaluated at $Q$, 
which is given by
\[l_{R,R}(Q): X_1^2 X_Q + T Y_Q + S.\]
where $R = (X_1:Y_1:Z_1)$ 
and $S=Y_1^2$ and $T=Z_1^2$ are the values 
that were computed in the point doubling computation.
Set $Q' = (X_{Q'}:Y_{Q'}:1)$ and $Q = \varphi(Q')$, where
$\varphi : \mathcal{H}_a \rightarrow \mathcal{H}_1$ 
is the twist isomorphism \eqref{twistiso}
(this is possible as $3|k$).
Then we can write $l_{R,R}(Q)$ as
\[l_{R,R}(Q): (S \cdot Y_{Q'} + T) + a X_{Q'} \cdot X_1^2 \omega,\]
which can be computed with cost $\sqr + \frac{2k}{3}\mul$ via
\[\begin{array}{ccc}
U = X_1^2; & V = S \cdot Y_{Q'}; & W = \eta \cdot U;
\end{array}\]
\[l_{R,R}(Q)= V+T + W\omega,\]
where $\eta = aX_{Q'}$ and can be precomputed.

We now split into cases.

\paragraph*{Even embedding degrees.}

By Section \ref{denelim}, we can set the denominator of the Miller doubling function to 1, so that the computation of the line function $l_{R,R}(Q)$ is in fact
the computation of the whole Miller (doubling) function $\ell_{R,R}(Q)$.

Furthermore, a general element of $\F{q^k}$ considered as element of the 
$\F{q^{k/3}}$-vector space generated
by $\omega$ will be of the form 
\[c_1 \omega + c_2 \omega^2 + c_3 \omega^3,\]
but for $\ell_{2R}(Q)$ we have that $c_2 = 0$. 
In particular, the multiplication of $\ell_{2R}(Q)$ with $f^2$ in
Step 3 of Algorithm~\ref{algo:miller} will not be the full cost of a general multiplication in $\F{q^k}$ (that
is, approximately $k^2\mul$), 
but by schoolbook multiplication will cost 6 multiplications in $\F{q^{k/3}}$, 
which amounts to 
$6 \left(\frac{k}{3}\right)^2 \mul = \frac{2}{3} \Mul$.
Putting together all of the above, the Miller doubling step for even
embedding degrees costs
\[\left( 5 + \frac{2k}{3}\right)\mul + 3\sqr + \frac{2}{3}\Mul + 1\Sqr.\]

\paragraph*{Odd embedding degrees.}

By Section \ref{denelim}, we have
\[\ell_{2R}(Q) = n_{2R}(Q) \cdot l_{R,R}(Q),\]
where $n_{2R}(Q)$ is as given in Section \ref{denelim}.
The multiplication of $n_{2R}(Q)$ with $l_{R,R}(Q)$ 
costs $\frac{2}{3}\Mul$ 
as $l_{R,R}(Q)$ has no coefficient of $\omega^2$.

Putting the above together, the Miller doubling step 
for odd embedding degrees costs
\begin{align*}
3\sqr + \left(5 + \frac{4k}{3}\right)\mul + \frac{11}{9}\Sqr + \frac{16}{9}\Mul.
\end{align*}

\subsubsection{Pairings on $\G{2} \times \G{1}$}

For pairings on $\G{2} \times \G{1}$ the input points are
$P \in \G{2}$ and $Q \in \G{1}$, and
$R$ will be a multiple of $P$.
We choose $P = (X_P:Y_P:1) \in \varphi(\mathcal{H}_a(\F{p^{k/3}}))$, where $\varphi$
is the twist isomorphism given in Equation \eqref{twistiso}.
As $R=(X_1,Y_1,Z_1)$ is a multiple of $P$ it is also in the image of $\mathcal{H}_a(\F{p^{k/3}})$ under $\varphi$; 
let $R' \in \mathcal{H}_a(\F{p^{k/3}})$ be the preimage of $R$ under $\varphi$.
As $2R = 2\varphi(R') = \varphi(2R')$, 
we can peforming the doubling operation on the cubic twist $\mathcal{H}_a$, 
so that the operation count occurs in $\F{q^{k/3}}$.
That is, point doubling can be performed in 5 multiplications and 2 squarings in $\F{q^{k/3}}$, which amounts to $\frac{5}{9}\Mul + \frac{2}{9}\Sqr$.
For even embedding degrees this can be done slightly faster, which we address below.

As for pairings on $\G{1} \times \G{2}$, 
we address the computations of the line function 
\begin{equation}\label{l2R}
l_{R,R}(Q): X_1^2 X_Q + T Y_Q + S,
\end{equation}
where $S = Y_1^2$ and $T = Z_1^2$,
in order to compute the Miller doubling function.

\paragraph*{Even embedding degrees.}

Assume now that $6|k$. 
As described in Section \ref{denelim}
we choose the input point from $\G{2}$, 
in this case $P = \varphi(P')$, such that $P'$ is
in the image of the quadratic twist isomorphism $\psi$ given in 
Section~\ref{sextic}. 
This implies that $R' = \varphi^{-1}(R)$, 
as a multiple of $P'$, also lies in this image, 
so that there exist $U_1, V_1, W_1 \in \F{q^{k/6}}$ for which
\begin{equation}\label{Rprime}
R' = (X_1':Y_1':Z_1') = (U_1:\beta(\alpha V_1 - 54W_1) : \beta(-\alpha V_1 + 54\zeta_3^2 W_1)),
\end{equation}
where $\beta = \zeta_3-\zeta_3^2$ and 
$\zeta_3 \in \F{q}$ is a primitive cube root of unity.
Here $\omega$ and $\alpha$ are as in Section~\ref{sextic}.
In particular we have that $X_1' \in \F{q^{k/6}}$ and
$Y_1',Z_1' \in \F{q^{k/3}}$.

This gives us a small saving in the point doubling calculation:
In the preamble we stated that all the point doubling arithmetic is performed 
in $\F{q^{k/3}}$. 
However, the final step in the computation of $X_3'$
(the $X$-coordinate of $2R'$) is not a full multiplication in $\F{q^{k/3}}$
but a multiplication of a $\F{q^{k/6}}$-element $X_1'$ with a 
$\F{q^{k/3}}$-element $(A-B)$,
costing $2\left(\frac{k}{6}\right)^2\mul = \frac{1}{18}\Mul$
using schoolbook multiplication instead of $\frac{1}{9}\Mul$.
So we save $\frac{1}{18}\Mul$ on the point doubling for even embedding degrees, giving an operation count of $\frac{1}{2}\Mul + \frac{2}{9}\Sqr$.

As shown in Section \ref{denelim}, the Miller doubling function $\ell_{R,R}(Q)$ is just given by the line function $l_{R,R}(Q)$ in this case, the computation of which we now address.
As above we have that $R = (X_1:Y_1:Z_1) = (X_1'\omega:Y_1':Z_1')$ 
so that Equation \eqref{l2R} becomes
\[l_{R,R}(Q): (X_1')^2X_Q\omega^2 + T Y_Q + S.\]
The values $S$ and $T$ are computed during the point doubling computation and lie in $\F{q^{k/3}}$,
so the computation of $\ell_{R,R}(Q) = l_{R,R}(Q)$ costs an additional squaring in $\F{q^{k/6}}$, multiplication of a $\F{q^{k/6}}$-element with a $\F{q}$-element, and multiplication of a $\F{q^{k/3}}$-element with a $\F{q}$-element, that is 
$\frac{1}{36}\Sqr + \frac{k}{2}\mul$, via
\[\begin{array}{ccc}
c_1 = (X_1')^2; & c_2 = c_1 \cdot  X_Q; & c_3 = T \cdot Y_Q.
\end{array}\]
Additionally, the formula for $\ell_{R,R}(Q)$ considered as an element of the $\F{q^{k/6}}$-vector space generated by $\omega$ and $\alpha$ has no coefficient of $\omega$, $\alpha\omega$, or $\alpha\omega^2$.
Therefore the multiplication of $\ell_{R,R}(Q)$ with a general element 
(i.e. $f^2$) of
$\F{q^k}$ costs only $3\cdot 6\left(\frac{k}{6}\right)^2\mul= \frac{1}{2}\Mul$ with schoolbook arithmetic.

Putting the above together, the full Miller doubling step for even embedding degrees costs
$\frac{k}{2}\mul + \Mul + \frac{7}{6}\Sqr.$

\paragraph*{Odd embedding degrees.}

By Section~\ref{denelim}, the Miller doubling function $\ell_{R,R}(Q)$ is given by
\[\ell_{R,R}(Q) = n_{2R}(Q) \cdot l_{R,R}(Q),\]
where $n_{2R}(Q)$ is as given in Section~\ref{denelim}.
As described above for even embedding degrees, we have that
\[l_{R,R}(Q): (X_1')^2X_Q\omega^2 + T Y_Q + S,\]
where $S,T \in \F{q^{k/3}}$ and are computed during the point doubling computation. 
In the case of odd embedding degrees, we have that $X_1' \in \F{q^{k/3}}$,
so that the cost of computating $l_{R,R}(Q)$ via $c_1$, $c_2$, and $c_3$ as above is
$\frac{1}{9}\Sqr + \frac{2k}{3}\mul$.
The multiplication of $l_{R,R}(Q)$ with $n_{2R}(Q)$ costs only $\frac{2}{3}\Mul$ as $l_{R,R}(Q)$ has no coefficient of $\omega$, so the total cost of the computation of $\ell_{R,R}(Q)$ is 
$\left(\frac{2k}{3}\mul + \frac{2}{9}\Sqr + \frac{1}{9}\Mul\right)
+ \left(\frac{2k}{3}\mul + \frac{1}{9}\Sqr \right)
+ \frac{2}{3}\Mul
=
\frac{1}{3}\Sqr + \frac{7}{9}\Mul + \frac{4k}{3}\mul$.

Putting the above together, the whole Miller doubling step costs
\[\frac{7}{3}\Mul + \frac{14}{9}\Sqr + \frac{4k}{3}\mul\]
for odd embedding degrees.

\subsection{Miller addition}

Let $P_1 = P = (X_1:Y_1:1)$ and $P_2 = R = (X_2:Y_2:Z_2) \in \mathcal{H}_b(K)$ for $b \in \{1,a\}$.
The fastest known formulas to compute $P_1 + P_2 = P_3 = (X_3:Y_3:Z_3)$
for $P_1 \neq P_2$ (due to \cite{2012/hess}) are as follows:
\begin{align*}
A &= X_1 \cdot Z_2;\	\qquad
C = Y_1 \cdot X_2;\	\qquad
D = Y_1 \cdot Y_2;\	\qquad
F = \eta \cdot X_2;\\
G &= (D + Z_2) \cdot (A - C);\	\quad
H = (D - Z_2) \cdot (A + C);\\
J &= (D + F) \cdot (A - Y_2);\	\quad
K = (D - F) \cdot (A + Y_2);\\
X_3 &= G - H;\	\qquad
Y_3 = K - J;\\
Z_3 &= J + K - G - H - 2(Z_2 - F) \cdot (C + Y_2),
\end{align*}
where $\eta = a\cdot X_1$ can be precomputed.
The cost for point addition with the above formulas is 9 multiplications in $K$.

\subsubsection{Pairings on $\G{1} \times \G{2}$}

Recall that the Miller addition function is given by
\[\ell_{P_1,P_2}(Q) = l_{P_1,P_2}(Q)/v_{P_1 + P_2}(Q).\]
For pairings on $\G{1} \times \G{2}$ the input points are $P \in \G{1}$ and
$Q \in \G{2}$, and for addition we have that
 $P_1  = P = (X_1:Y_1:1)$ and $P_2 = R = (X_2:Y_2:Z_2)$ is a multiple of $P$.
 
The line $l_{P_1,P_2}(Q)$ is the line passing through $P$ and $R$ evaluated at $Q$.
As above we write $Q = (\omega X_Q':Y_Q':1)$ with
$Q' = (X_{Q'},Y_{Q'}:1) \in \mathcal{H}_a(\F{q^{k/3}})$.
Then
$$l_{P,R}(Q): (E - Y_2) \cdot X_1 + (Y_{Q'} - Y_1) \cdot (A - X_2) - (E - Y_2) \cdot X_{Q'}\omega,  $$
where $E = Y_1 \cdot Z_2$, and where $A$ is the value that was computed during the computation of $P+R$.
In particular, the cost of computing $l_{P,R}(Q)$ is $\left(2 + \frac{2k}{3}\right)\mul$ via
\[\begin{array}{ccc}
E = Y_1 \cdot Z_2; & L = (E-Y_2)\cdot X_1; 
& M = (Y_{Q'} - Y_1) \cdot (A - X_2); 
\end{array}\]
\[\begin{array}{cc}
N = (E - Y_2) \cdot X_{Q'}; &
l_{P,R}(Q) = L + M - N\omega.
\end{array}\]

\paragraph*{Even embedding degrees.}

By Section~\ref{denelim}, the Miller addition function $\ell_{P_1,P_2}(Q)$ is just given by $l_{P_1,P_2}(Q)$ in this case. 
Also, exactly as for the the Miller doubling function, multiplying a general element of $\F{q^k}$ with $l_{P,R}(Q)$ costs only
$\frac{2}{3}\Mul$.
Putting together all of the above, the entire Miller addition step costs
\[\left(11 + \frac{2k}{3}\right)\mul + \frac{2}{3}\Mul.\]

\paragraph*{Odd embedding degrees.}

By Section~\ref{denelim}, the Miller doubling function $\ell_{P_1,P_2}(Q)$ is given by
\[\ell_{P_1,P_2}(Q) = n_{P_1+P_2}(Q) \cdot l_{P_1,P_2}(Q),\]
where $n_{P_1+P_2}(Q)$ is as given in Section~\ref{denelim}.
The multiplication of $n_{P_1+P_2}$ with $l_{P_1,P_2}(Q)$ costs
$\frac{2}{3}\Mul$ as $l_{P_1,P_2}(Q)$ has no coefficient of $\omega^2$,
so the computation of $\ell_{P_1,P_2}(Q)$ in this case costs
\[\left( \frac{2k}{3}\mul + \frac{2}{9}\Sqr + \frac{1}{9}\Mul\right) 
+\left(2 + \frac{2k}{3}\right)\mul 
+\frac{2}{3}\Mul
=\left(2 + \frac{4k}{3}\right)\mul + \frac{2}{9}\Sqr + \frac{7}{9}\Mul.\]
Putting together all of the above, the entire Miller addition step costs
\[\left(11 + \frac{4k}{3}\right)\mul + \frac{2}{9}\Sqr + \frac{16}{9}\Mul.\]

\subsubsection{Pairings on $\G{2} \times \G{1}$}

For pairings on $\G{2} \times \G{1}$ the input points $P \in \G{2}$ and $Q \in \G{1}$, and in the Miller addition function $\ell_{P_1,P_2}(Q)$ we have that
$P_1 = P = (X_1:Y_1:1)$ and $P_2 = R = (X_2:Y_2:Z_2)$, which is some multiple of $P$. 
In exactly the same way as discussed for the Miller doubling function,
the point addition can be performed in the group $\mathcal{H}_a(\F{q^{k/3}})$
in place of $\mathcal{H}(\F{q^k})$, so that the operation count occurs in
$\F{q^{k/3}})$. That is, point addition can be performed in 9 multiplications in 
$\F{q^{k/3}}$, which amounts to $1\Mul$. 
For even embedding degrees this can be done faster, which we adress below.
As for pairings on $\G{1} \times \G{2}$, we will need to compute the line function
$$l_{P,R}(Q): - (E - Y_2) \cdot X_{Q} + (E - Y_2) \cdot X_1 + (Y_{Q}-Y_1)(A-X_2), $$
where $E = Y_1 \cdot Z_2$ and $A = X_1 \cdot Z_2$.
Let $P = \varphi(P')$ and $R = \varphi(R')$ be the images of 
$P' = (X_1',Y_1',1)$ and $R' = (X_2',Y_2',Z_2') \in \mathcal{H}_a(\F{q^{k/3}})$ respectively
under the twist isomorphism $\varphi$ of Equation \eqref{twistiso}.
Then 
$$l_{P,R}(Q): - (E' - Y_2') \cdot X_{Q} + (C'-Y_2'X_1' + Y_Q(A'-X_2'))\omega, $$
where $E' = Y_1' \cdot Z_2'$, $A = A'\omega$, $C = C'\omega$ and 
$A' = X_1'Z_2'$ and $C' = Y_1'X_2'$ are the values that were computed during the point addition.
This can be computed in 
$\frac{2}{9}\Mul + \frac{2k}{3}\mul$
via
\[\begin{array}{cccc}
E' = Y_1' \cdot Z_2'; &
d_1 = Y_2' \cdot X_1'; &
d_2 = (E' - Y_2') \cdot X_Q; &
d_3 = (A' - X_2')\cdot Y_Q.
\end{array}\]

\paragraph*{Even embedding degrees.}

Suppose now that $6 | k$.
As described already for Miller doubling, we may choose $U_2,V_2,W_2 \in \F{q^{k/6}}$ such that
\[ R' = (U_2:\beta(\alpha V_2 - 54W_2) : \beta(-\alpha V_2 + 54\zeta_3^2 W_2)),
\]
where $\beta = \zeta_3-\zeta_3^2$ and 
$\zeta_3 \in \F{q}$ is a primitive cube root of unity (c.f. Equation \eqref{Rprime}). 
Note that we do not apply this to $P$ because we want to make use of the mixed addition with $Z_1 = 1$.

This gives us a small saving in the point addition calculation: the computations of $C$ and of $F$ now cost $\frac{1}{18}\Mul$ each instead of $\frac{1}{9}\Mul$ each, saving $\frac{1}{18}\Mul$; 
the cost for point addition is therefore $\frac{17}{18}\Mul$.

As shown in Section~\ref{denelim}, the Miller addition function $\ell_{P,R}(Q)$
is just given by the line function $l_{P,R}(Q)$ in this case. Multiplication of a general element in $\F{q^k}$ with $\ell_{P,R}(Q)$ costs only $\frac{2}{3}\Mul$
as $\ell_{P,R}(Q)$ has no coefficient of $\omega^2$.

Putting together all of the above, we get an operation count of
\[\frac{2k}{3}\mul + \frac{11}{6}\Mul \]
for the whole Miller addition step.

\paragraph*{Odd embedding degrees.}

By Section~\ref{denelim}, the Miller addition function 
$\ell_{P,R}(Q)$ is given by $n_{P+R}(Q) \cdot l_{P,R}(Q)$ in this case, where
$n_{P+R}(Q)$ is as given in Section~\ref{denelim}.
The multiplication of $n_{P+R}(Q)$ with $l_{P,R}(Q)$ costs only $\frac{2}{3}\Mul$ as $l_{P,R}(Q)$ has no coefficient of $\omega^2$.

Putting together all of the above, we get an operation count of
\[\frac{4k}{3}\mul + 3\Mul + \frac{2}{9}\Sqr\]
for the full Miller addition step.
