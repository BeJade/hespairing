\section{Computation of line functions}
\label{sec:lines}

The efficiency of pairings substantially relies on the computation of the line functions
(as denoted by $\ell_{2R} = l_{2R}/v_{2R}$ and $\ell_{R,P} = l_{R,P}/v_{R,P}$ in Algorithm~\ref{algo:miller}).
Recall that an equation of a line is simply $y = mx + b$.
Therefore, constructing the line $l$ passing through points $P$ and $R$
is done by substituting the coordinates $x$ and $y$ of points $P$ and $R$
then solving for $m$ and $b$.
Once the line is constructed, it is then evaluated at another point $Q$

Specifically, to compute the line $l_{R,P}$ passing through
points $R = (x_R, y_R)$ and $P = (x_P, y_P)$ where $R \ne P$
and evaluated at a point $Q = (x_Q, y_Q)$,
we perform the following computations:
%let $P = (x_P, y_P)$ and  $R = (x_R, y_R)$ where $P \ne R$.
%Let $Q = (x_Q, y_Q)$ be the point to evaluate.
%The line $l_{R,P}$ evaluated at $Q$ is constructed as follows.
\begin{align*}
m_{l_{R,P}} &= (y_P - y_R)/(x_P - x_R),			\\
b_{l_{R,P}} &= y_P - m_{l_{R,P}} \cdot x_P,		\\
l_{R,P} &= y_Q - (m_{l_{R,P}} \cdot x_Q + b_{l_{R,P}}). 
\end{align*}
Note that %for this case where $P \ne R$,
the line passing through points $P$ and $R$ is also passing through point $-(P+R)$.
Using any two out of these three points would result in the same line $l$.

To compute the line $l_{2R}$ tangent at the point $R$,
we use the fact that this line also intersects the curve at the point $-(2R)$.
Therefore, we construct a line passing through the points $R$ and $-(2R)$.
Recall that for curves in twisted Hessian forms,
a negation of a point $2R = (x_{2R}, y_{2R})$ is $-2R = (x_{2R}/y_{2R}, 1/y_{2R})$.
Let $R = (x_{R}, y_{R})$
and $-2R = (x_{2R}/y_{2R}, 1/y_{2R})$. % = (x'_{2R},y'_{2R})$.
The line $l$ tangent at the point $R$ evaluated at the point $Q$ is computed as follows:
\begin{align*}
m_{l_{2R}} &= \bigg( \frac{1}{y_{2R}} - x_{R} \bigg) / \bigg( \frac{x_{2R}}{y_{2R}} - x_{R} \bigg),	\\
b_{l_{2R}} &= y_R - m_{l_{2R}} \cdot x_R,		\\
l_{2R} &= y_Q - (m_{l_{2R}} \cdot x_Q + b_{l_{2R}}).
\end{align*}

To construct the line $v$ passing through points $(P+R)$ and its negation $-(P+R)$,
recall that for Weierstrass curves,
this line $v$ is simply a vertical line
and thus of the form $y = c$.
However, for the (twisted) Hessian curves, the line $v$ is not a vertical line
and thus has to be constructed in a similar way as the line $l$,
i.e., solving for the $m$ and $b$ of $y = mx + b$ using points $P+R$ and $-(P+R)$.
Note also that the line passing through the point and its negation also passes through
the point $(0,-1)$, the neutral point of twisted Hessian forms.
% (0 : -1 : 1)

Let $(P+R) = (x_{RP},y_{RP})$.
By using the neutral point $(0,-1)$,
the line $v_{R,P}$ is computed as follows:
\begin{align*}
m_{v_{R,P}} &= (y_{RP} + 1) / x_{RP},	\\
b_{v_{R,P}} &= -1,			\\
v_{R,P} &= y_Q - (m_{v_{R,P}} \cdot x_Q + b_{v_{R,P}}).
\end{align*}
The advantage of using the neutral point is that
it saves some computations and $b_{v_{R,P}}$ becomes $-1$
due to a multiplication by zero.
For the line $v_{2R}$,
the calculation is similar to the line $v_{R,P}$,
namely,
simply replace $(P+R)$ by $(2R) = (x_{2R}, y_{2R})$
and compute as above.

Recall that the lines $v_{R,P}$ and $v_{2R}$ appear as denominators,
namely, we have to compute $l_{2R}/v_{2R}$ and $l_{R,P}/v_{R,P}$.
%Because computing inversion is expensive,
%we apply ``denominotr elimination'' to optimize this inversion.
Note that for even embedding degree,
this line $v$ lies in a subfield of $\F{q^k}$
which becomes $1$ after computing the final exponentiation
({line 8} in Algorithm~\ref{algo:miller} in Section~\ref{sec:background}).
Therefore, for even embedding degrees, the line $v$ can be omitted.
However, for odd embedding degrees, we cannot neglect the denominator $v$.
Because computing the inversion is expensive,
we first apply the ``denominator elimination'' technique to optimize this inversion.

%%%%%%%%%%%%%%%%%%%%%%%%%%%%%%%%%%%%%%%%%%%%%%%%%%%%%%%%%%%%%%%%%%%%%%%%%%%%%%%%%%%%%%%%%%%%%%%%%%%%

%\subsection{Denominator elimination}
We follow a similar technique as described in~\cite{2008/lin} and~\cite{2009/deg15}
by rewriting an inversion into a fraction for which the denominator lies in a subfield.
Define a field extension $\F{q^k}$ of a finite field $\F{q}$ and suppose that $x,y \in \F{q^k}$.

Observe that
$$ \frac{1}{x-y} = \frac{x^2 + xy + y^2}{x^3 - y^3}. $$
As $x^3$ and $y^3$ lie in a subfield of $\F{q^k}$, we have that
\[(x^{3} - y^{3})^{\frac{q^{k}-1}{r}} = 1.\]
Hence, after the final exponentiation in the computation of the pairing, this factor does not appear.
This means that we can ignore the computation of the denominator $x^3 - y^3$.
Therefore, instead of computing $\frac{1}{x-y}$, we compute $x^2 + xy + y^2$.
The cost of division by $x - y$ then becomes the cost of multiplication by $x^2 + xy + y^2$.

%%%%%%%%%%%%%%%%%%%%%%%%%%%%%%%%%%%%%%%%%%%%%%%%%%%%%%%%%%%%%%%%%%%%%%%%%%%%%%%%%%%%%%%%%%%%%%%%%%%%

Note that in the previously presented formulas and explanations,
we used points represented in affine coordinates,
e.g., $P = (x,y)$.
However, to avoid inversion,
we use points represented in projective coordinates,
e.g., $P = (X : Y : Z)$ where $x = X/Z$ and $y = Y/Z$.
The formulas given in the following section use projective coordinates.
