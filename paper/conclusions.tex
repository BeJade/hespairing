\section{Concluding remarks}
\label{sec:conclude}

This paper presents efficient formulas to compute Miller doubling and Miller addition on curves of $j$-invariant 0 with embedding degree divisible by 3 when written in Hessian form.
This paper presents formulas for both pairings of the form $\G{1} \times \G{2}$ and $\G{2} \times \G{1}$ and compares the efficiency of these formulas to the best known formulas of previous research.
We present the first formulas for pairings on $\G{2} \times \G{1}$ that utilize twists of degree 3 in the case of odd embedding degrees, and the first formulas that utilize twists of degree 3 for Hessian curves in all cases. 
Table~\ref{tbl-cmp1} and the discussion of the previous sections 
shows that for embedding degrees 15, 21, and (possibly) 24 
our formulas are the most efficient among known choices.

Curves generated by the methods used in this paper (originally due to~\cite{2010/freeman}) are guaranteed to have twists of degree $3$
and have embedding degree
$k \equiv 3,9,15 \pmod{18}$ or
$k \equiv 0 \pmod{6}$ where $18 \nmid k$.
As explained in earlier sections, we suggest updating the use of embedding degree 12 to 15 for 128-bit security and 18 to 21 for 192-bit security in light of the NFS attacks and their variants.
This allows us to keep the relatively small primes for the base field and a low $\rho$-value. 
We additionally suggest including $k=24$ in any future (more precise) comparisons, 
as Table~\ref{tbl-cmp1} shows that this may be competitive with $k=21$
%(Note also in the families mentioned is this paper,
(since the $\rho$-value for $k=24$ is lower than that of $k=21$).

In future work,
we plan to study precisely how the NFS attacks and their variants apply to our constructions in order to be able to
properly evaluate the security and propose concrete parameters.
A comparison between the larger embedding degrees (but low $\rho$-value) that we suggest
in this paper and the higher $\rho$-value (but small embedding degrees) suggested in~\cite{2018/FK} would be very interesting, but we leave this for future work.
It would also be interesting to evaluate the performance of other curve models with degree 3 twists on $\G{2} \times \G{1}$ pairings.
We also consider the optimized implementation as future work.

%Our formulas for even embedding degrees are the fastest among all the previously published one. - No they're not (twisted Edwards are faster)
%We also describe techniques to eliminate intermediate denominators for odd embedding degrees. - this isn't new

%Although pairings on Hessian curves have already been considered,
%for example, by Gu, Gu, and Xie in~\cite{2010/Gu} and by Li and Zhang in~\cite{2012/Li},
%no methods regarding how to generate pairing-friendly Hessian curves were described.

%The pairing-friendly families of BN, BLS, and KSS curves use primarily embedding degrees 12 or 18.
%Due to recent advances in the discrete logarithm problem, it has become necessary to increase the size of the finite field,
%which can be done either by increasing the size of the prime or by increasing the embedding degree.
%Our constructions address this problem by generating pairing-friendly twisted Hessian curves with embedding degrees 15 and 21,
%which allows for an increase in security without having to jump from embedding degree 12 to 18, or from 18 to 24.

%%XXX emphasize advantages not shown in the tables
%There are various advantages of our constructions that cannot be justified by the cost shown in Table~\ref{tbl-cmp}. %Table~\ref{tbl-even} and Table~\ref{tbl-odd}.
%First of all, the costs shown in those tables do not reflect the total cost for the pairing-based protocols.
%This is because generically there are many group operations performed prior to relatively few pairing computations,
%and group operations on twisted Hessian curves allow faster point arithmetic operations than Weierstrass curves.
%Secondly, having curves represented in the same models for $\mathbb{G}_1$ and $\mathbb{G}_2$ does not induce the extra cost of conversion between curve models.
%(Recall that for BN, BLS, and KSS, this conversion is always necessary if one wants to take advantage of
%the fast point-arithmetic on Hessian or Edwards curves, as proven in~\cite{2013/bos-pairing}.)

%We also would like to emphasize that embedding degrees such as 12, 18, 24 can split into many subfields.
%Recall that recent advances in number field sieve attacks on solving DLP~\cite{2016/KB} which further generalized by~\cite{2017/KJ} and~\cite{2016/conjugate}
%%~\cite{2016/SS}
%target on finite field $\F{q^n}$ where $q$ is prime and $n$ is composite.
%This implies that the attacks apply to pairings which use composite extension degrees.
%
%%XXX Laurent said not convincing !!!
%To be more precise, these attacks rely on splitting $n$ into $n = \eta \kappa$ for non-trivial factorization.
%Since, for example, 12 can split as $(\eta,\kappa) = (2,6), (3,4), (4,3), (6,2)$,
%it allows many possibles choices of subfields to perform the attacks.
%Similar remarks also apply to
%$18$ which can be split as $(2,9), (3,6), (6,3), (9,2)$
%and
%$k = 24$ which can be split as $(2,12)$, $(3,8)$, $(4,6)$, $(6,4)$, $(8,3)$, $(12,2)$.
%%Note that some methods above all so have a restriction of $gcd(\eta,\kappa) = 1$, but some of the do not force this condition.
%
%On the other hand, there are sufficiently fewer subfields in our case.
%For example, 15 splits into either $(3,5)$ or $(5,3)$
%while 21 splits into either $(3,7)$ or $(7,3)$.
%This means that we reduce the possible choices of subfields that the attacks can target.

%In this paper, we do not propose any concrete parameters for our pairing-friendly twisted Hessian curves.
%This is due to the fact that the complexity of the new attacks have neither been completely understood nor reached the stable stage yet.
%Moreover, there is no analysis of how those attacks would apply to embedding degree 15 or 21.

