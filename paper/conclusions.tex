\section{Concluding remarks}
\label{sec:conclude}

This paper presents concrete methods to generate pairing-friendly twisted Hessian curves.
Curves generated by these methods are guaranteed to have twists of degree $3$
and have embedding degree
$k \equiv 3,9,15 \pmod{18}$ or
$k \equiv 0 \pmod{6}$ where $18 \nmid k$.
We describe techniques to eliminate intermediate denominators for odd embedding degrees
and to compute the optimal ate pairing entirely on the twisted curves.
%which leading to faster overall computation.
We also provide explicit formulas to compute line functions in pairing computations.

Although pairings on Hessian curves have already been considered,
for example, by Gu, Gu, and Xie in~\cite{2010/Gu} and by Li and Zhang in~\cite{2012/Li},
this work is the first, to our knowledge, to concretely explain methods to generate
pairing-friendly ``twisted'' Hessian curves with odd embedding degree.
Twisted Hessian curves are a generalization of Hessian curves; using twists allows our formulas to be applied to more curves.
We also take advantage of the state-of-the-art fast point arithmetic formulas available~\cite{2015/hessian}
on curves in twisted Hessian form.

The pairing-friendly families of BN, BLS, and KSS curves use primarily embedding degrees 12 or 18.
Due to recent advances in the discrete logarithm problem, it has become necessary to increase the size of the finite field,
which can be done either by increasing the size of the prime or by increasing the embedding degree.
Our constructions address this problem by generating pairing-friendly twisted Hessian curves with embedding degrees 15 and 21,
which allow a better tuning instead of having to jump from embedding degree 12 to 18, or from 18 to 24.

We also would like to emphasize that embedding degrees such as 12, 18, 24 can split into many subfields.
Recall that recent advances in number field sieve attacks on solving DLP~\cite{2016/conjugate}~\cite{2016/SS}~\cite{2016/KB}~\cite{2017/KJ},
target on finite field $\F{q^n}$ where $q$ is prime and $n$ is composite.
This implies that the attacks apply to pairings which use composite extension degrees.

To be more precise, these attacks rely on splitting $n$ into $n = \eta \kappa$ for non-trivial factorization.
Since, for example, 12 can split as $(\eta,\kappa) = (2,6), (3,4), (4,3), (6,2)$,
it allows many possibles choices of subfields to perform the attacks.
Similar remarks also apply to
$18$ which can be split as $(2,9), (3,6), (6,3), (9,2)$
and
$k = 24$ which can be split as $(2,12)$, $(3,8)$, $(4,6)$, $(6,4)$, $(8,3)$, $(12,2)$.
%Note that some methods above all so have a restriction of $gcd(\eta,\kappa) = 1$, but some of the do not force this condition.

On the other hand, there are sufficiently fewer subfields in our case.
For example, 15 splits into either $(3,5)$ or $(5,3)$
while 21 splits into either $(3,7)$ or $(7,3)$.
This means that we reduce the possible choices of subfields that the attacks can target.

In this paper, we do not propose any concrete parameters for our pairing-friendly twisted Hessian curves.
This is due to the fact that the complexity of the new attacks have neither been completely understood nor reached the stable stage yet.
Moreover, there is no analysis of how those attacks would apply to embedding degree 15 or 21.
We plan to study how the attacks apply to our constructions in order to be able to
evaluate the security level and propose concrete parameters
We also consider the optimized implementation as future work.
