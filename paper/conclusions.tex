\section{Concluding remarks}
\label{sec:conclude}

%XXX
% ours is for TWISTED Hessian curves
% which is more general than Gu, Gu and Xie which is only Hessian curves
% our formulas are also faster than them

This paper presents concrete methods to generate pairing-friendly twisted Hessian curves.
Curves generated by these methods are guaranteed to have twists of degree $3$, and
have embedding degree
$k \equiv 3 \pmod{18}$,
$k \equiv 9,15 \pmod{18}$ or
$k \equiv 0 \pmod{6}$ where $18 \nmid k$.
We describe techniques to eliminate intermediate denominators for odd embedding degrees, leading to faster overall computation.
We also provide explicit formulas to compute line functions in pairing computations.
%Our constructions of pairing-friendly twsited Hessian curves together with presented techniques and formulas
%offer fast point arithmetic and fast line function computations
%competitive with pairing-friendly Weierstrass and Edwards curves.

Although pairings on Hessian curves have already been considered,
for example, by Gu, Gu, and Xie in~\cite{2010/Gu} and by Li and Zhang in~\cite{2012/Li},
this work is the first to explain methods to generate pairing-friendly ``twisted'' Hessian curves with odd embedding degree, to our knowledge. Twisted Hessian curves are a generalization of Hessian curves; using twists allows our formulas to be applied to more curves. We also take advantage of the state-of-the-art fast point arithmetic formulas from~\cite{2015/hessian} available on curves in Hessian form.
%and present faster formulas for line function and pairing computations.

The pairing-friendly families of BN, BLS, and KSS curves use primarily embedding degrees 12 or 18. Due to recent advances in the discrete logarithm problem, it has become necessary to increase the size of the finite field, which can be done either by increasing the size of the prime or by increasing the embedding degree. Our constructions address this problem by generating pairing-friendly twisted Hessian curves with embedding degrees 15 and 21.

%Our initial analysis shows that the speed of line function computations on Hessian curves
%is comparable to those on Weierstrass curves.
%However, Hessian curves have an advantage of faster point arithmetic,
%which may lead to faster overall performance of pairing computations.
%%In constrast to the widely used Weierstrass curves,
%%Hessian curves provide faster point arithmetic formulas.
%%With our constructions of pairing-friendly Hessian curves
%%together with all presented techniques related to pairing computations on Hessian curves,
%We expect pairing on Hessian curves to be faster than on Weierstrass curves and
%consider the optimized implementation as future work.
