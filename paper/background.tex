\section{Background on Pairings}
\label{sec:background}

Let $E$ be an elliptic curve defined over a finite field $\FF_{q}$ where $q$ is a prime.
Let $r$ be the largest prime factor of $n = \#E(\FF_{q}) = q + 1 - t$ where $t$ is a trace of Frobenius.
The \emph{embedding degree} with respect to $r$ is defined to be the smallest positive integer $k$ such that $r | (q^k-1)$.
Let $\mu_{r} \subseteq \FF_{q^{k}}^{*}$ be the group of $r$-th roots of unity.
For $m \in \ZZ$ and $P \in E[r]$,
let $f_{m,P}$ be a function with divisor 
$\text{div}(f_{m,P}) = m(P) - ([m]P) - (m-1)(\mathcal{O})$,
where $\mathcal{O}$ denotes the point at infinity on $E$.
The reduced Tate pairing is defined as
\[\begin{array}{rccl}
\tau_{r} :	& E(\FF_{q^{k}})[r] \times E(\FF_{q^{k}})/[r]E(\FF_{q^{k}})	& \longrightarrow	& \mu_{r} \\
		& (P,Q)								& \mapsto		& f_{r,P}(Q)^{\frac{q^{k}-1}{r}}.
\end{array}\]
We address the computation of the reduced Tate pairing restricted to $\G{1} \times \G{2}$, where
$$\G{1} = E[r] \cap \ker(\phi_{q}-[1]) \text{ and }
\G{2} = E[r] \cap \ker(\phi_{q} - [q]) \subseteq E(\FF_{q^{k}}).$$
Here $\phi_{q}$ denotes the $q$-power Frobenius morphism on $E$.
We denote the restriction of $\tau_{r}$ to $\G{1} \times \G{2}$ by
$$e_{r} : \G{1} \times \G{2} \longrightarrow \mu_{r}.$$
Let $T = t-1$.
By restricted the Tate pairing to $\G{2} \times \G{1}$, the ate pairing is defined as
\[\begin{array}{rccl}
a_{T} :	& \G{2} \times \G{1}	& \longrightarrow	& \mu_{r} \\
	& (Q,P) 		& \mapsto		& f_{T,Q}(P)^{\frac{q^{k}-1}{r}}.
\end{array}\]
Note that in addition to the arguments $P$ and $Q$ being switched, the parameter $r$ is also changed to $T$.

Miller's algorithm for computing pairings works as follows.
Let $m$ be an integer expressed in a binary format as $(m_{\ell-1},\dots,m_1,m_0)_2$.
First, a rational function $f_{m,P}$ associated to a point $P$ and $m$ is constructed.
Then, this function is evaluated at a point $Q$.
Miller's algorithm computes $f_{m,P}$ in the double-and-add manner
and uses the following relation
$$ f_{i+j,P} = f_{i,P} f_{j,P} \frac{l}{v}, $$
where $l$ is a line through points $iP$ and $jP$,
and $v$ is a line through points $(i+j)P$ and $-(i+j)P$.

Constructing the line $l$ can be done by substituting $x$ and $y$ coordinates of points $iP$ and $jP$
into the equation $y = mx + b$ and solving for $m$ and $b$.
For Weierstrass curves, the line $v$ is simply a vertical line through $(i+j)P$.
Therefor, the line $v$ is of the form $x = c$ where $c$ is the $x$-coordinate of the point $(i+j)P$.
Note that the points $(i+j)P$ and $-(i+j)P$ have the same $x$-coordinate.
For (twisted) Hessian curves, however, the line $v$ is not a vertical line
and thus has to be constructed in a similar way as the line $l$,
i.e., solving for $m$ and $b$ of $y = mx + b$ using points $(i+j)P$ and $-(i+j)P$.

Algorithm~\ref{algo:miller} shows the Miller's algorithm.
We denote $\ell_{2R} = l_{2R} / v_{2R}$ where $l_{2R}$ is the tangent line at point $R$
while $v_{2R}$ is the line passing through points $2R$ and its negation $-2R$.
We denote $\ell_{R,P} = l_{R,P} / v_{R,P}$ where $l_{R,P}$ is the line passing through points $R$ and $P$
while $v_{R,P}$ is the line passing through points $R+P$ and its negation $-(R+P)$.

\begin{algorithm}
\caption{Miller's algorithm}
\label{algo:miller}
\begin{algorithmic}[1]

	\Require $m = (m_{\ell-1}, \dots, m_1, m_0)_2$ and $P,Q \in E[r]$ with $P \ne Q$
	\State Initialize $R = P$ and $f = 1$

	\For {$i := \ell - 2$ {\bf{down to}} $0$}
		\State $f \leftarrow f^2 \cdot \ell_{2R}(Q)$
		\State $R \leftarrow 2R$
		\If {$m_i = 1$}
			\State $f \leftarrow f \cdot \ell_{R,P}(Q)$
			\State $R \leftarrow R+P$
		\EndIf
	\EndFor
	\State $f \leftarrow f^{(q^k-1)/r}$

\end{algorithmic}
\end{algorithm}

