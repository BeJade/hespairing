\section{Background on Pairings}
\label{sec:background}

Let $E$ be an elliptic curve defined over a finite field $\FF_{q}$ where $q$ is a prime.
Let $r$ be the largest prime factor of $n = \#E(\FF_{q}) = q + 1 - t$ where $t$ is the trace of Frobenius.
The \emph{embedding degree} with respect to $r$ is defined to be the smallest positive integer $k$ such that $r | (q^k-1)$.
Let $\mu_{r} \subseteq \FF_{q^{k}}^{*}$ be the group of $r$-th roots of unity.
For $m \in \ZZ$ and $P \in E[r]$,
let $f_{m,P}$ be a function with divisor 
$\text{div}(f_{m,P}) = m(P) - ([m]P) - (m-1)(\mathcal{O})$,
where $\mathcal{O}$ denotes the neutral element of $E$.
The reduced Tate pairing is defined as
\[\begin{array}{rccl}
\tau_{r} :	& E(\FF_{q^{k}})[r] \times E(\FF_{q^{k}})/[r]E(\FF_{q^{k}})	& \longrightarrow	& \mu_{r} \\
		& (P,Q)								& \mapsto		& f_{r,P}(Q)^{\frac{q^{k}-1}{r}}.
\end{array}\]
We address the computation of the reduced Tate pairing restricted to $\G{1} \times \G{2}$, where
$$\G{1} = E[r] \cap \ker(\phi_{q}-[1]) \text{ and }
\G{2} = E[r] \cap \ker(\phi_{q} - [q]) \subseteq E(\FF_{q^{k}}).$$
Here $\phi_{q}$ denotes the $q$-power Frobenius morphism on $E$.
We denote the restriction of $\tau_{r}$ to $\G{1} \times \G{2}$ by
$$e_{r} : \G{1} \times \G{2} \longrightarrow \mu_{r}.$$
Let $T = t-1$.
We define the \emph{ate pairing} $a_{T}$ by restricting the Tate pairing to $\G{2} \times \G{1}$ so that
\[\begin{array}{rccl}
a_{T} :	& \G{2} \times \G{1}	& \longrightarrow	& \mu_{r} \\
	& (P,Q) 		& \mapsto		& f_{T,P}(Q)^{\frac{q^{k}-1}{r}}.
\end{array}\]
Note that in addition to $\G{1}$ and $\G{2}$ being switched, the subscript $r$ (i.e., the number of loops) is also changed to $T$.

%Miller's algorithm for computing pairings works as follows.
%Let $m$ be an integer expressed in a binary format as $(m_{n-1},\dots,m_1,m_0)_2$.
%First, a rational function $f_{m,P}$ associated to a point $P$ and to $m$ is constructed.
%Then, this function is evaluated at a point $Q$.
%Miller's algorithm computes $f_{m,P}$ in the double-and-add manner
%and uses the following relation
%$$ f_{i+j,P} = f_{i,P} f_{j,P} \frac{l}{v}, $$
%where $l$ is a line through points $iP$ and $jP$,
%and $v$ is a line through points $(i+j)P$ and $-(i+j)P$.

%Constructing the line $l$ can be done by substituting $x$ and $y$ coordinates of points $iP$ and $jP$
%into the equation $y = mx + b$ and solving for $m$ and $b$.
%For Weierstrass curves, the line $v$ is simply a vertical line through $(i+j)P$.
%Therefore, the line $v$ is of the form $x = c$ where $c$ is the $x$-coordinate of the point $(i+j)P$.
%Note that the points $(i+j)P$ and $-(i+j)P$ have the same $x$-coordinate.
%For (twisted) Hessian curves, however, the line $v$ is not a vertical line
%and thus has to be constructed in a similar way as the line $l$,
%i.e., solving for $m$ and $b$ of $y = mx + b$ using points $(i+j)P$ and $-(i+j)P$.

Algorithm~\ref{algo:miller} shows Miller's algorithm to compute the reduced Tate pairing or the ate pairing. 
Let $m \in \{r,T\}$ and represent the binary format of $m$ by $(m_{n-1},\dots,m_1,m_0)_2$.
For any two points $R,S$ on $E$ denote by $l_{R,S}$ the line passing through
$R$ and $S$, and by $v_R$ the line passing through $R$ and $-R$.
We further define $\ell_{2R} = l_{R,R}/v_{2R}$ and
$\ell_{R,P} = l_{R,P} / v_{R+P}$.
Miller's algorithm outputs the Tate pairing if $m = r$, $P \in \G{1}$, and $Q \in \G{2}$, and outputs the ate pairing if $m = T$, $P \in \G{2}$, and $Q \in \G{1}$.

\begin{algorithm}
\caption{Miller's algorithm}
\label{algo:miller}
\begin{algorithmic}[1]

	\Require $m = (m_{n-1}, \dots, m_1, m_0)_2$ and $P,Q \in E[r]$ with $P \ne Q$
	\State Initialize $R = P$ and $f = 1$

	\For {$i := n - 2$ {\bf{down to}} $0$}
		\State $f \leftarrow f^2 \cdot \ell_{2R}(Q)$
		\State $R \leftarrow 2R$
		\If {$m_i = 1$}
			\State $f \leftarrow f \cdot \ell_{R,P}(Q)$
			\State $R \leftarrow R+P$
		\EndIf
	\EndFor
	\State $f \leftarrow f^{(q^k-1)/r}$

\end{algorithmic}
\end{algorithm}

