\section{Introduction}
\label{sec:intro}

Pairings on elliptic curves have various applications in cryptography,
ranging from very basic key exchange protocols,
such as one round tripartite Diffie--Hellman~\cite{2000/joux-ants}~\cite{2004/joux-tripartite},
to complicated protocols, such as
identity-based encryption~\cite{2001/boneh}~\cite{2002/horwitz}~\cite{2002/gentry}~\cite{2005/sahai}.
Pairings also help to improve currently existing protocols, 
such as signature schemes, to have shortest possible signatures~\cite{2004/boneh}.

Curves that are suitable for pairings are called {\emph{pairing-friendly curves}},
and these curves must satisfy specific properties.
It is extremely rare that any randomly generated elliptic curves result in pairing-friendly curves.
Therefore, pairing-friendly curves need to be generated in a special way.
Examples of famous and commonly used pairing-friendly curves include 
Barreto-Naehrig curves~\cite{2006/barreto} (BN curves),
Barreto-Lynn-Scott curves~\cite{2003/bls} (BLS curves), and
Kachisa-Schaefer-Scott curves~\cite{2008/kss} (KSS curves).

Performance of pairing-based cryptography relies on
elliptic-curve-point arithmetic, computation of line functions and pairing algorithms.
A pairing is a bilinear map from two elliptic curve groups $\mathbb{G}_1$ and $\mathbb{G}_2$ to
a target group $\mathbb{G}_T$.
%$e: \mathbb{G}_1 \times \mathbb{G}_2 \rightarrow \mathbb{G}_T$.
%where $\mathbb{G}_1, \mathbb{G}_2$ and $\mathbb{G}_T$ are groups of prime order $r$.
Therefore, to achieve a good performance,
it is desirable to have a fast elliptic-curve-point arithmetic in both $\mathbb{G}_1$ and $\mathbb{G}_2$.
Speeding up the computation of line functions and pairing algorithms also help improving the overall performance of pairings.

Security of pairings depends on the cost of solving discrete logarithm problem (DLP) in the three groups previously mentioned,
namely, $\mathbb{G}_1$, $\mathbb{G}_2$, and $\mathbb{G}_T$.
Since one can attack pairing-based protocols by attacking any of these three groups,
the cost of solving DLP must be sufficiently high in all of these three groups.
Recall that $\mathbb{G}_1$ and $\mathbb{G}_2$ are elliptic-curve groups
while $\mathbb{G}_T$ is a multiplicative group defined over a finite field.
In general, DLP on the elliptic-curve side is more difficult than the finite field side.
This means that the group size of the finite field has to be larger in order to have a balance security on both sides.

%%%%%%%%%%%%%%%%%%%%%%%%%%%%%%%%%%%%%%%%%%%%%%%%%%%%%%%%%%%%%%%%%%%%%%%%%%%%%%%%%%%%%%%%%%%%%%%%%%%%

\subsection{Choice of curves and embedding degrees}

One way to improve the performance of pairings is
to improve the performance of the underlying point arithmetic.
Since pairing computation arises naturally from a geometric representation of Weierstrass curves,
most pairing formulas consider Weierstrass curves.
However, arithmetic on Weierstrass curves is known to be quite slow compare to other curves,
%Alternative curve models other than the conservative Weierstrass curves have been proposed,
for example, Hessian curves~\cite{2001/smart}~\cite{2001/joye} and
Edwards curves~\cite{2007/edwards}~\cite{2007/bernstein-newelliptic}.

Pairings based on Edwards curves
along with examples of pairing-friendly Edwards curves
were proposed by Arene, Lange, Naehrig and Ritzenthaler~\cite{2009/fastertate}.
They found that the computation of line functions is much more complicated in the situation of Edwards curves than Weierstrass curves.
%since formulas to compute line function derive naturally from Weierstrass model.
In other words,
%the gain due to the fast point arithmetic on Edwards curves is outweighed by the loss of slow computation of line functions.
even though Edwards curves allow faster point arithmetic,
this gain is outweighed by the slower computation of line functions.
%the computation of line funcions is much slower which outweigh
%%leands to a slowdown in the overall computation of pairing.

Pairings based on Hessian curves have been considered by Gu, Gu and Xie~\cite{2010/Gu}.
They provided a geometric interpretation of the group law on Hessian curves
along with an algorithm for computing Tate pairing on elliptic curves in Hessian form.
However, no pairing-friendly curves in the Hessian forms were given.

Bos, Costello and Naehrig~\cite{2013/bos-pairing} investigated a possibility of
using faster curve models to compute operations prior to pairings, e.g., group exponentiations,
then mapping to Weierstrass curves only for pairing computations.
They found that among BN-12, BLS-12, and KSS-18 families of pairing-friendly curves
either $\mathbb{G}_1$ or $\mathbb{G}_2$ has to be represented in Weierstrass curves.
This means that either $\mathbb{G}_1$ or $\mathbb{G}_2$ has to use slower formulas to compute point operations.
Moreover, this idea of using different curve models comes at a cost of at least one conversion
between other curve models into Weierstrass curves.

Regardless of the curve models, most of the previous works were considering ``even'' embedding degrees.
One of the main advantages of even embedding degrees is the applicability of applying a denominator elimination technique.
Originally, part of the pairing algorithms is to compute $l \leftarrow l_1/l_2$.
In the case of even embedding degrees, $l_2$ will be mapped to one in a sequent step (final exponentiation).
Therefore, a computation of $l_2$ and a division by $l_2$ can completely be ignored.
On the other hands, $l_2$ will not be mapped to one for odd embedding degrees.
Thus, this denominator elimination technique cannot be applied directly.

Despite the disadvantage of having to compute $l_2$,
pairings with odd embedding degrees have been considered before.
A different technique of the denominator elimination for odd embedding degrees has been explained by
Lin, Zhao, Zhang and Wang~\cite{2008/lin} which considered embedding degree 9 and by
Mrabet, Guillermin and Ionica~\cite{2009/deg15} which considered embedding degree 15.
A recent article by Fouotsa, Mrabet and Pecha~\cite{2016/degodd} considered embedding degrees 9, 15, and 27.
However, these works concern pairings on conventional Weierstrass curves.

%%%%%%%%%%%%%%%%%%%%%%%%%%%%%%%%%%%%%%%%%%%%%%%%%%%%%%%%%%%%%%%%%%%%%%%%%%%%%%%%%%%%%%%%%%%%%%%%%%%%

\subsection{Attacks on solving DLP over finite fields}

The series of advance in computing discrete logarithm in finite field,
i.e.,~\cite{2016/KB}~\cite{2015/BGGM}~\cite{2015/BGK},
have weakened the security of pairings by dramatically decreased the security level on the finite field side.
In order to retain the security level, either the size of the prime field or the embedding degree (or both) has to be increased.
For example, BN-12 (BN curves with embedding degree 12) used to be very popular for 128-bit security level
because 256-bit prime field maps to 3072-bit finite field
where the security on both prime field side and finite field side match 128-bit security level.
%However, series of advancement in computing discrete logarithm in finite field, for instance,
%which have dramatically decreased the security level on the finite field side.
A recent article on ``Updating key size estimations for parings'' by Barbulescu and Duquesne~\cite{2017/keysize}
suggests to increase the prime field to be over 400 bits for BN curves in order to achieve 128-bit security level.
Increasing the field size has a penalty of slowing down arithmetic operations which creates a huge impact on performance.

%%%%%%%%%%%%%%%%%%%%%%%%%%%%%%%%%%%%%%%%%%%%%%%%%%%%%%%%%%%%%%%%%%%%%%%%%%%%%%%%%%%%%%%%%%%%%%%%%%%%

\subsection{Our contributions}

%xxx To Chloe: I don't know how to put these in this subsection.  Note that the attacks also apply to odd embedding degrees.  In general, they apply to all composite embedding degrees.
%However, recent advances in the number field sieve techniques (\cite{?})
%to attack the discrete logarithm problem for elliptic curves have significantly reduced the security of pairing-based cryptography
%for elliptic curves with even embedding degree.
%In this paper, we outline compute the reduced Tate pairing on elliptic curves in Hessian form,
%which have a natural degree 3 twist, so that the most natural embedding degrees to consider are multiples of 3.

In terms of the performance of pairing-based protocols,
the paper~\cite{2013/bos-pairing} by Bos, Costello and Naehrig
inspires the question whether it is possible to
construct pairing-friendly curves that could be represented in the same curve models,
preferably with fast point arithmetic and fast computation of line functions,
for both $\mathbb{G}_1$ and $\mathbb{G}_2$,

Twisted Hessian curves with faster point arithmetic have recently been proposed by
Bernstein, Lange, Chuengsatiansup and Kohel~\cite{2015/hessian}.
This leads to questions whether the newer point arithmetic would also lead to faster formulas for computing line functions,
and whether these formulas would be applicable to pairing on twisted Hessian curves.

We observed that computing line functions on twisted Hessian curves is very similar to computing line functions on Weierstrass curves.
We investigated methods~\cite{2010/freeman} to generate pairing-friendly curves
and propose concrete constructions to generate pairing-friendly curves
that can be represented in the twisted Hessian form for both $\mathbb{G}_1$ and $\mathbb{G}_2$.
Pairing-friendly twisted Hessian curves generated by these constructions allow twist of degree 3.
We also derived formulas for computing pairing on twisted Hessian curves based on the state-of-the-art point arithmetic formulas.

Note that point arithmetic operations on twisted Hessian curves are faster than on Weierstrass curves.
Therefore, we gain advantages of faster operations prior to pairing computations.
Moreover, thanks to the fact that both $\mathbb{G}_1$ and $\mathbb{G}_2$ are already in the same curve model,
we do not have to perform any conversion for the curve shapes.

Recall that most of the previous works focused on Weierstrass curves and even embedding degrees.
There were some previous works on different curve models but with even embedding degrees.
There were also some previous works with odd embedding degrees but on Weierstrass models.
We appear to be the first to consider pairings on curves other than Weierstrass models,
specifically ``twisted'' Hessian curves, with ``odd'' embedding degrees.

Among the popular pairing-friendly curve families,
BN-12 and KSS-18 have embedding degrees 12 and 18 respectively.
We notice that jumping from embedding degree 12 to 18 is quite a huge gap.
Therefore, we provide an alternative of embedding degree 15 aiming at a better tune of prime field and embedding degree.
For higher security level, our constructions also allow, for example, embedding degrees 21 and 33.

%XXX add reference to section
The setup of this article is as follows:
in ? we detail how to compute the reduced Tate pairing on an elliptic curve in Hessian form, making use of the existence of degree 3 twists.
In ?, we observe that the efficiency of computing line functions on curves in Hessian form is similar to the Weierstrass case.
In ?, we recall the methods given in \cite{2010/freeman} to generate pairing-friendly curves
and propose a method to generate high-security pairing-friendly curves
that can be represented in the Hessian form.
In ?, we state explicit formulae for the computation of the reduced Tate pairing on Hessian curves based on the state-of-the-art point arithmetic formulas.
In ?, we compare the efficiency of our algorithm with the Weierstrass case.

%%%%%%%%%%%%%%%%%%%%%%%%%%%%%%%%%%%%%5

