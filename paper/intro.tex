\section{Introduction}
\label{sec:intro}

Pairings on elliptic curves have various applications in cryptography,
ranging from very basic key exchange protocols,
such as one round tripartite Diffie--Hellman~\cite{2000/joux-ants}~\cite{2004/joux-tripartite},
to complicated protocols, such as
identity-based encryption~\cite{2001/boneh}~\cite{2002/horwitz}~\cite{2002/gentry}~\cite{2005/sahai}.
Pairings also help to improve currently existing protocols, 
such as signature schemes, to have shortest possible signatures~\cite{2004/boneh}.

Curves that are suitable for pairings are called {\emph{pairing-friendly curves}},
and these curves must satisfy specific properties.
It is extremely rare that a randomly generated elliptic curve is pairing-friendly,
so pairing-friendly curves have to be generated in a specific way.
Examples of famous and commonly used pairing-friendly curves include 
Barreto-Naehrig curves~\cite{2006/barreto} (BN curves),
Barreto-Lynn-Scott curves~\cite{2003/bls} (BLS curves), and
Kachisa-Schaefer-Scott curves~\cite{2008/kss} (KSS curves).

The performance of pairing-based cryptography relies on
elliptic-curve-point arithmetic, computation of line functions and pairing algorithms.
A pairing is a bilinear map from two elliptic curve groups $\mathbb{G}_1$ and $\mathbb{G}_2$ to
a target group $\mathbb{G}_T$.
To achieve a good performance, as well as having an efficient pairing algorithm,
it is also desirable to have a fast elliptic-curve-point arithmetic in both $\mathbb{G}_1$ and $\mathbb{G}_2$.

The security of pairings depends mainly on the cost of solving the discrete logarithm problem (DLP) in the three groups previously mentioned,
namely, $\mathbb{G}_1$, $\mathbb{G}_2$, and $\mathbb{G}_T$.
Since one can attack pairing-based protocols by attacking any of these three groups,
the cost of solving DLP must be sufficiently high in all of these three groups.

%%%%%%%%%%%%%%%%%%%%%%%%%%%%%%%%%%%%%%%%%%%%%%%%%%%%%%%%%%%%%%%%%%%%%%%%%%%%%%%%%%%%%%%%%%%%%%%%%%%%

\subsection{Choice of curves and embedding degrees}

One way to improve the performance of pairings is
to improve the performance of the underlying point arithmetic.
Many authors have studied efficient point arithmetic via the 
representation of elliptic curves in a specific model,
for example, Hessian form~\cite{2001/smart}~\cite{2001/joye} or
Edwards form~\cite{2007/edwards}~\cite{2007/bernstein-newelliptic}.

Pairings based on Edwards curves,
along with examples of pairing-friendly Edwards curves,
were proposed by Arene, Lange, Naehrig and Ritzenthaler~\cite{2009/fastertate}.
They found that the computation of line functions necessary to compute the pairing
is much more complicated than if the curves were written in Weierstrass form.
%since formulas to compute line function derive naturally from Weierstrass model.
In other words,
%the gain due to the fast point arithmetic on Edwards curves is outweighed by the loss of slow computation of line functions.
even though Edwards curves allow faster point arithmetic,
this gain is somewhat outweighed by the slower computation of line functions.
Li, Wu, and Zhang~\cite{2014/LWZ} proposed the use of quartic and sextic twists
for Edwards curves, improving the efficiency of both the point arithmetic and
the computation of the line functions.
%the computation of line functions is much slower which outweigh
%%leads to a slowdown in the overall computation of pairing.

Pairings based on Hessian curves with even embedding degrees were proposed by Gu, Gu and Xie~\cite{2010/Gu}.
They provided a geometric interpretation of the group law on Hessian curves
along with an algorithm for computing Tate pairing on elliptic curves in Hessian form.
However, no pairing-friendly curves in Hessian form were given.

Bos, Costello and Naehrig~\cite{2013/bos-pairing} investigated the possibility of
using a model of a curve (such as Edwards or Hessian) allowing for fast point arithmetic
and transforming to Weierstrass form for the actual computation of the pairing.
They found that for every elliptic curve $E$ in the BN-12, BLS-12, and KSS-18 families of pairing-friendly curves,
if $E$ is isomorphic over $\F{q}$ to a curve in Hessian or Edwards form,
then it is not isomorphic over $\F{q^{k}}$ to a curve in Hessian or Edwards form,
where $k$ is the embedding degree.
This implies that the point arithmetic has to be performed on curves in Weierstrass form -- not all curves can be written in special forms such as Hessian or Edwards form.
This idea of using different curve models comes at a cost of at least one conversion
between other curve models into Weierstrass form.

In this article we study the efficiency of curves in Hessian form for pairing computations. 
Hessian curves with $j$-invariant 0 have degree-3 twists that can also be written in Hessian form. This
means that we can take full advantage of speed-up techniques for
point arithmetic and pairing computations that move arithmetic to subfields
via the twist, for example as studied for Edwards curves in~\cite{2014/LWZ},
without the expensive curve conversion to Weierstrass form.
We use the families proposed by~\cite{2010/freeman}, in which we could find
three families that can be written in Hessian form.

Regardless of which model of elliptic curve was being studied,
most of the previous articles on this topic were considering even embedding degrees.
One of the main advantages of even embedding degrees is the applicability of a denominator elimination technique in the pairing computation 
(avoiding a field inversion)
which does not directly apply to odd embedding degrees.
Examples of pairing algorithms for curves in Weierstrass form with odd embedding degree include
the work by Lin, Zhao, Zhang and Wang in~\cite{2008/lin}, by Mrabet, Guillermin and Ionica in~\cite{2009/deg15},
and by Fouotsa, Mrabet and Pecha in~\cite{2016/degodd}.
%In this paper, we concentrate on curves with embedding degrees divisible by 3,
%as pairing-friendly elliptic curves in (twisted) Hessian form are equipped with natural twists of degree 3.

\subsection{Attacks on solving DLP over finite fields}

Due to recent advances in number field sieve techniques for attacking the discrete logarithm problem for pairing-friendly elliptic curves over finite fields~\cite{2013/jouxP}~\cite{2016/KB}~\cite{2015/BGGM}~\cite{2015/BGK} (NFS attacks),
it is necessary to re-evaluate the security of pairing-friendly curves.
In~\cite{2018/FK}, Fotiadis and Konstantinou propose countering these attacks by using families with a higher $\rho$-value. 
In this paper, we investigate the feasability of an alternative method: increasing the embedding degree.
This has the advantage of keeping the low $\rho$-value of previously proposed families, but it is disadvantaged by the less efficient pairing computations.
This article attempts to analyze the use of Hessian curves in combatting this.
Previous research on computing pairings with  Hessian curves addressed only even emebedding degrees, and in order to make use of degree-3 twists the embedding degree should be divisible by 6. 
Prior to the NFS attacks and their variants, the favoured embedding degree for 128-bit security was 12, 
so that to increase the embedding degree while making use of cubic twists the next
candidate is 15. However, as 15 is odd the formulas of~\cite{2010/Gu} do not apply;
for this reason one focus of this article is to provide formulas for embedding degree 15. Simlarly, the pre-NFS favourite embedding degree for 192-bit security was 18, which we propose to increase to 21.
Observe further that for 192-bit security, the families of~\cite{2018/FK} all require the embedding degree to be greater than $21$.
%This means increasing the conventional even embedding degree 12 to 18 or to 24 for higher security levels (see below).

%Originally, part of the pairing algorithms is to compute $l \leftarrow l_1/l_2$.
%In the case of even embedding degrees, $l_2$ will be mapped to one in a sequent step (final exponentiation).
%Therefore, a computation of $l_2$ and a division by $l_2$ can completely be ignored.
%On the other hands, $l_2$ will not be mapped to one for odd embedding degrees.
%Thus, this denominator elimination technique cannot be applied directly.

%Despite the disadvantage of having to compute $l_2$,
%pairings with odd embedding degrees have been considered before.
%A different technique of the denominator elimination for odd embedding degrees has been explained by
%Lin, Zhao, Zhang and Wang~\cite{2008/lin} which considered embedding degree 9 and by
%5Mrabet, Guillermin and Ionica~\cite{2009/deg15} which considered embedding degree 15.
%A recent article by Fouotsa, Mrabet and Pecha~\cite{2016/degodd} considered embedding degrees 9, 15, and 27.
%However, these works concern pairings on conventional Weierstrass curves.

%%%%%%%%%%%%%%%%%%%%%%%%%%%%%%%%%%%%%%%%%%%%%%%%%%%%%%%%%%%%%%%%%%%%%%%%%%%%%%%%%%%%%%%%%%%%%%%%%%%%


%%%%%%%%%%%%%%%%%%%%%%%%%%%%%%%%%%%%%%%%%%%%%%%%%%%%%%%%%%%%%%%%%%%%%%%%%%%%%%%%%%%%%%%%%%%%%%%%%%%%

\subsection{Our contributions}

We present formulas for computing pairings on both $\G{1} \times \G{2}$ and $\G{2} \times \G{1}$ for a curve given in Hessian form that admits degree-3 twists.
These formulas exploit the degree-3 twists where possible: in moving the point arithmetic in $\F{q^k}$ to $\F{q^{k/3}}$ and performing the computations for the line functions in $\F{q^{k/3}}$ in place of $\F{q^k}$.
For efficient curve arithmetic (before applying the use of twists) we refer to
Bernstein, Chuengsatiansup, Kohel, and Lange~\cite{2015/hessian}.

We analyze the efficiency of the pairing computation in each case. 
Our analysis shows that for even embedding degrees, Hessian curves are outperformed by twisted Edwards curves for pairings on $\G{1} \times \G{2}$, and by Weierstrass curves for pairings on $\G{2} \times \G{1}$.
However, as explained above, our focus is on odd embedding degrees, as we propose 
the use of $k=15$ and $k=21$ as a countermeasure against the NFS attacks and their variants.
For embedding degrees 15 and 21 we show that the most efficient pairing
computation available is the optimal ate pairing with the Hessian form proposed in this paper.

We also give concrete constructions of pairing-friendly Hessian curves for both embedding degrees and a proof-of-concept implementation of the optimal ate pairing for these cases. 
%Note in particular that among the popular pairing-friendly curve families,
%BN-12 and KSS-18 have embedding degrees 12 and 18 respectively, which gives a large difference in field size.
%Our results provide an alternative of embedding degree 15,
%in an attempt to give a better balance between the size of the prime field and of the embedding degree.
%For higher security levels, our constructions allow, for example, embedding degrees 21 or 33.

The setup of this paper is as follows:
in Section~\ref{sec:background}, we recall Miller's algorithm to compute pairings on elliptic curves.
In Section~\ref{sec:curves}, we state constructions of families of pairing-friendly elliptic curves given in~\cite{2010/freeman}
that can be written in twisted Hessian form, together with the conversion
to twisted Hessian form.
%In Section~\ref{sec:lines}, we observe that the efficiency of computing line functions on curves in twisted Hessian form is similar to the Weierstrass case
%and explain how to compute them.
In Section~\ref{sec:lines}, we provide explicit formulas for the computation of 
pairings for both $\G{1} \times \G{2}$ and $\G{2} \times \G{1}$ 
on twisted Hessian curves
using the state-of-the-art point arithmetic formulas presented in~\cite{2015/hessian}.
%In Section~\ref{sec:even}, we state explicit formulas for the computation of the optimal ate pairing
%on twisted Hessian curves having even embedding degrees based on the state-of-the-art point arithmetic formulas.
%We also provide comparisons to previous works.
%In Section~\ref{sec:odd}, we give explicit formulas for the odd embedding degree case
%and a comparison.
In Section~\ref{sec:cmp}, we compare our results to previous work.
In Section~\ref{sec:conclude}, we present our conclusions.
%and remark on some advantages that do not obviously appear in the comparison tables.
%In Section~\ref{sec:formulas}, we state explicit formulas for the computation of the optimal ate pairing
%on twisted Hessian curves based on the state-of-the-art point arithmetic formulas.
%In Appendix~\ref{app:magma}, we provide full Magma codes for a toy example of pairing-friendly twisted Hessian curves with embedding degree $k = 21$.

%%%%%%%%%%%%%%%%%%%%%%%%%%%%%%%%%%%%%5

