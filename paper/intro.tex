%%\documentclass[11pt, a4paper]{article}
%\documentclass[11pt, a4paper]{llncs}
%
%\usepackage[a4paper]{geometry}
%\usepackage{a4wide}
%
%%\usepackage{a4wide}
%%\usepackage[margin=3.8cm]{geometry}
%
%\usepackage{xcolor}
%\definecolor{linkcolor}{rgb}{0,0,0.8}
%\definecolor{citecolor}{rgb}{0,0,0.8}
%\definecolor{urlcolor}{rgb}{0,0,0.8}
%\usepackage[colorlinks=true, linkcolor=linkcolor, urlcolor=urlcolor, citecolor=citecolor]{hyperref}
%\usepackage{breakurl}
%\usepackage{url}
%\usepackage{hyperref}
%
%\usepackage{amsfonts}
%
%\begin{document}
%
%\title{Fast Pairings on Pairing-Friendly Hessian Curves}
%
%\author{Chitchanok Chuengsatiansup}
%
%\institute{
%Department of Mathematics and Computer Science\\
%Technische Universiteit Eindhoven\\
%P.O. Box 513, 5600 MB Eindhoven, The Netherlands\\
%\email{c.chuengsatiansup@tue.nl}
%}
%
%%\date{}
%%\maketitle

\section{Introduction}

Pairings on elliptic curves have various applications in cryptography,
ranging from very basic key exchange protocols,
such as one round tripartite Diffie--Hellman~\cite{2000/joux-ants}~\cite{2004/joux-tripartite},
to complicated protocols, such as
identity-based encryption~\cite{2001/boneh}~\cite{2002/horwitz}~\cite{2002/gentry}~\cite{2005/sahai}.
Pairings also help to improve currently existing protocols, 
such as signature schemes to have short signatures~\cite{2004/boneh}.

Curves that are suitable for pairings are called {\emph{pairing-friendly curves}},
and these curves must satisfy specific properties.
It is extremely rare that any randomly generated elliptic curves result in pairing-friendly curves.
Therefore, pairing-friendly curves need to be generated in a special way.
Examples of famous and commonly used pairing-friendly curves include 
Barreto-Naehrig curves~\cite{2006/barreto} (BN-curves),
Barreto-Lynn-Scott curves~\cite{2003/bls} (BLS-curves), and
Kachisa-Schaefer-Scott curves~\cite{2008/kss} (KSS-curves).

Performance of pairing-based cryptography relies on both
elliptic-curve-point arithmetic and a computation of line functions.
A pairing is a bilinear map from two elliptic curve groups $\mathbb{G}_1$ and $\mathbb{G}_2$ to
a target group $\mathbb{G}_T$.
%$e: \mathbb{G}_1 \times \mathbb{G}_2 \rightarrow \mathbb{G}_T$.
%where $\mathbb{G}_1, \mathbb{G}_2$ and $\mathbb{G}_T$ are groups of prime order $r$.
Therefore, it is desirable to have efficient arithmetic in both $\mathbb{G}_1$ and $\mathbb{G}_2$.
To improve the performance of the underlying point arithmetic,
alternative curve models other than the conservative Weierstrass curves have been proposed,
for example, Hessian curves~\cite{2001/smart}~\cite{2001/joye} and
Edwards curves~\cite{2007/edwards}~\cite{2007/bernstein-newelliptic}.

Pairings based on Edwards curves
along with examples of pairing-friendly Edwards curves
were proposed by Arene, Lange, Naehrig and Ritzenthaler~\cite{2009/fastertate}.
They found that the computation of line functions is much more complicated in the situation of Edwards curves than Weierstrass curves.
%since formulas to compute line function derive naturally from Weierstrass model.
In other words,
%the gain due to the fast point arithmetic on Edwards curves is outweighed by the loss of slow computation of line functions.
even though Edwards curves allow faster point arithmetic,
this gain is outweighed by the slower computation of line functions.
%the computation of line funcions is much slower which outweigh
%%leands to a slowdown in the overall computation of pairing.

Bos, Costello and Naehrig~\cite{2013/bos-pairing} investigated a possibility of
using faster curve models to compute operations prior to pairings, e.g., group exponentiations,
then mapping to Weierstrass curves only for pairing computations.
They found that among BN-12, BLS-12, and KSS-18 families of pairing-friendly curves
either $\mathbb{G}_1$ or $\mathbb{G}_2$ has to be represented in Weierstrass curves.
This means that either $\mathbb{G}_1$ or $\mathbb{G}_2$ has to use slower formulas to compute point operations.
Moreover, this idea of using different curve models comes at a cost of at least one conversion
between other curve models into Weierstrass curves.

The above raises the question whether it is possible to construct pairing-friendly curves that could be
represented in other curve models in both $\mathbb{G}_1$ and $\mathbb{G}_2$
and preferably with fast point arithmetic and fast computation of line functions.
Twisted Hessian curves with faster point arithmetic have recently been proposed by
Bernstein, Lange, Chuengsatiansup and Kohel~\cite{2015/hessian}.
This leads to questions whether the newer point arithmetic would also lead to faster formulas for computing line functions,
and whether these formulas would be applicable to pairing on Hessian curves.

We have observed that computing line functions on Hessian curves is very similar to computing line functions on Weierstrass curves.
We have investigated methods~\cite{2010/freeman} to generate pairing-friendly curves
and proposed a method to generate high-security pairing-friendly curves
that can be represented in the Hessian form for both $\mathbb{G}_1$ and $\mathbb{G}_2$.
We also derived formulas for computing pairing on Hessian curves based on the state-of-the-art point arithmetic formulas.
Our new formulas are faster than previously proposed ones.
Moreover, our pairing-friendly Hessian curves achieve better performance than the conservative pairing on Weierstrass curves
or with the technique of mapping between different curve models.



%\bibliographystyle{plain}
%\bibliography{ref}

%\end{document}
