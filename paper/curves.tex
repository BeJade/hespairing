\section{Pairing-friendly twisted Hessian curves}

Recall that in Section \ref{Tate}, we defined
\begin{itemize}
\item $p$ to be prime,
\item $E$ to be an elliptic curve over $\FF_{p}$,
\item $r$ to be the size of the largest prime order subgroup of $E(\FF_{p})$, and
\item $k$ to be smallest integer for which $r | p^{k}-1$.
\end{itemize}
Under these assumptions, we also defined 
\[\begin{array}{ccc}
G_{1} = E[r] \cap \ker(\phi_{p} - [1]) & \text{and} & G_{2} = E[r] \cap \ker(\phi_{p} - [p]) \subseteq E(\FF_{p}^{k}),
\end{array}\]
and the reduced Tate pairing restricted to $G_{1} \times G_{2}$ was given by
\[\begin{array}{rccc}
e_{r}:& G_{1} \times G_{2} & \longrightarrow & \mu_{r} \\
& (P,Q) & \mapsto & f_{r,P}(Q)^{\frac{p^{k}-1}{r}}.
\end{array}\]
We will refer to an elliptic curve $E/\FF_{p}$ where $p,r,k$ are known and satisfy the above properties as \emph{pairing-friendly}. To construct the reduced Tate pairing, we search for pairing-friendly elliptic curves with a point of order 3, as these can be written in Hessian form. There are many constructions of parametric families of pairing-friendly curves listed in \cite{2010/freeman}. We recall below those families for which the curves also have a point of order 3, as shown in \cite[Section 5]{2015/hessian}.

%\subsection{Cyclotomic families}
%~\cite{2010/freeman}
%Moreover, the search was restricted to the curves such that twists of those curves (see section~\ref{twist})
%must also contain points of order 3.
%%to reduce the computation by working with twists of curves (see Section~\ref{twist}),
%%we focused on finding curves which allow their twists also have points of order 3.
%%In other words, curves that we found 
%%This means that those curves that we found are also expressible in the Hessian form.
%Having the original curves and their twists contain points of order 3,
%these curves are guaranteed to be expressible in the Hessian form.
%%we check for
%%- have point of order 3
%%- twist s also have point of order 3
%%- have q prime
%%- q and r prime 


%%%%%%%%%%%%%%%%%%%%%%%%%%%%%%%%%%%%%%%%%%%%%%%%%%%%%%%%%%%%%%%%%%%%%%%%%%%%%%%%%%%%%%%%%%%%%%%%%%%%

%Let $k$ be a positive integer with $k \le 1000$ and $18 \nmid k$.
\subsection{Construction 1: $k \equiv 3 \pmod{18}$}
%$k \equiv 3 \pmod{6}$ and
This construction follows Construction 6.6 in~\cite{2010/freeman}
under the first subcase where $k \equiv 3 \pmod{6}$.
Define
\begin{align*}
r(x) &= \Phi_{2k}(x),	\\
k(x) &= x^{k/3+1} + 1,	\\
q(x) &= \frac{1}{3} (x^2 - x + 1) (x^{2k/3} - x^{k/3} + 1) + x^{k/3+1},
\end{align*}
where $\Phi_{2k}(x)$ denotes the cyclotomic polynomial of degree $2k$. For infinitely many $x_{0}\in \mathbb{Z}$, we can construct an elliptic curve $E/\FF_{p(x_{0})}$ such that the integers $r=r(x_{0})$, $k=k(x_{0})$, and $p=p(x_{0})$ satisfy the properties for $E$ to be pairing-friendly. Furthermore, for such $x_{0}$, we have that $k(x_{0}) \equiv 3 \pmod{18}$.
%try $k = 21$ $x = 3*2^5 + 2$
%have point of order 3, also on twist, have q prime, but not both q and r  prime
%$D = 3 -> d = 3$ can't have d=6 because doesn't divide k
%Because the discriminant $D = 3$ and $6 \nmid k$,
%the only possible twist is of degree $d = 3$.


\subsection{Construction 2: $k \equiv 9,15 \pmod{18}$}
%If $k \equiv 3 \pmod{6}$ and
This construction follows Construction 6.6 in~\cite{2010/freeman}
under the second subcase where $k \equiv 3 \pmod{6}$.
Define
\begin{align*}
r(x) &= \Phi_{2k}(x),	\\
t(x) &= -x^{k/3+1} + x + 1,	\\
q(x) &= \frac{1}{3} (x+1)^2 (x^{2k/3} - x^{k/3} + 1) - x^{2k/3+1},
\end{align*}
where $\Phi_{2k}(x)$ denotes the cyclotomic polynomial of degree $2k$. For infinitely many $x_{0}\in \mathbb{Z}$, we can construct an elliptic curve $E/\FF_{p(x_{0})}$ such that the integers $r=r(x_{0})$, $k=k(x_{0})$, and $p=p(x_{0})$ satisfy the properties for $E$ to be pairing-friendly. Furthermore, for such $x_{0}$, we have that
$k(x_{0}) \equiv 9,15 \pmod{18}$.
%try $k = 15, 33$
%$ 15: x = 3*2^{31} + 2$
%$ 33: 3*2^{10} + 2$
%have point or order 3, also on twist, have q prime, but not both q and r prime
%$D = 3 -> d = 3$ can't have d=6 because does't divide k
%Similar to the previous construction,
%this construction only allows twist of degree $3$.


\subsection{Construction 3: $k \equiv 0 \pmod{6}$ and $18 \nmid k$}
This construction follows the last case of Construction 6.6 in~\cite{2010/freeman}.
Define
\begin{align*}
r(x) &= \Phi_k(x),	\\
t(x) &= x+1,		\\
q(x) &= \frac{1}{3} (x-1)^2 (x^{k/3} - x^{k/6} + 1) + x,
\end{align*}
where $\Phi_{k}(x)$ denotes the cyclotomic polynomial of degree $k$. For infinitely many $x_{0}\in \mathbb{Z}$, we can construct an elliptic curve $E/\FF_{p(x_{0})}$ such that the integers $r=r(x_{0})$, $k=k(x_{0})$, and $p=p(x_{0})$ satisfy the properties for $E$ to be pairing-friendly. Furthermore, for such $x_{0}$, we have that
$k(x_{0}) \equiv 0 \pmod{6}$, where $18 \nmid k(x_{0})$.
%try $k = 12$
%$x = 3*2^{63}$
%have point of order 3, also on twist, have q prime, have q and r prime
%$D = 3 -> d = 3,6$
%Even though this construction allows curves with embedding degree $6$,
%only curves with embedding degree $3$ can be expressed in Hessian for both
%the original curves and their twists.

%For $k = 15, 21, 33$, the only possible twist is cubic twist.
%For $k = 12$ even though twist of degree $2, 3, 4, 6$ are possible, 
%only cubic twist that allow the twist of curve to also be expressible in Hessian form.

%%%%%%%%%%%%%%%%%%%%%%%%%%%%%%%%%%%%%%%%%%%%%%%%%%%%%%%%%%%%%%%%%%%%%%%%%%%%%%%%%%%%%%%%%%%%%%%%%%%%

%\subsection{Twists of curves}
\subsection{Twists of degree $3$}
\label{twist}
%To reduce the cost of pairing computation, we considered the concept of twists of curves.
Let $E$ and $E'$ be elliptic curves over $\F{p}$.
We call $E'$ a \emph{twist} of $E$ if $E$ and $E'$ are isomorphic over some field extension of $\F{p}$.
More precisely, $E'$ is a {\emph{degree-$d$ twist}} of $E$ if they are isomorphic over a degree $d$ extension and not over any smaller field. We will see below that if $E$ is an elliptic curve over $\FF_{p}$ where $p \equiv 1 \text{ mod } 3$ and can be written in the form
\[E_{a}: aX^{3} + Y^{3} + Z^{3} = 0\]
then $E_{a}$ has twists of degree 3. We will also see that if the embedding degree $k$ is divisible by 3, the existence of these twists allow us to assume without loss of generality that 2 of the 3 coordinates of the point $Q \in E(\FF_{q^{k}})$ are in $\FF_{q^{k/3}}$, reducing the cost of the computation.

\begin{lemma}
Suppose that
\[E: aX^{3} + Y^{3} + Z^{3} = 0\] is an elliptic curve defined over a field $\FF_{p}$ with embedding degree $k$, that $p \equiv 1 \text{ mod } 3$, and that $3|k$. Then up to $\FF_{p^{k/3}}$-isomorphism there are exactly two degree 3 twists of $E$ defined over $\FF_{p^{k/3}}$. These twists are given by
\[E_{a'}: a'X^{3} + Y^{3} + Z^{3} = 0,\] for $a \neq a' \in \FF_{p^{k/3}}^{*}/(\FF_{p^{k/3}}^{*})^{3}$.
\end{lemma}

\begin{proof}
Every supersingular elliptic curve with a point of order 3 is of the form
\[E': \alpha X^{3} + \beta Y^{3} + \gamma Z^{3} = 0,\] where $\alpha,\beta,\gamma \in \FF_{p^{k/3}}$. Furthermore, we know that $\alpha$, $\beta$, and $\gamma$ are non-zero as $E'$ is non-singular. Hence every twist of $E$ is isomorphic over $\FF_{p^{k/3}}$ to an elliptic curve $E'$ of the above form with $\alpha,\beta,\gamma \in \FF_{p^{k/3}}^{*}/(\FF_{p^{k/3}}^{*})^{3}$. Let $a'$ be a generator of the group $\FF_{p^{k/3}}^{*}/(\FF_{p^{k/3}}^{*})^{3}$, so that
\[\FF_{p^{k/3}}^{*}/(\FF_{p^{k/3}}^{*})^{3} = \{1,a',a'^{2}\}.\] The curves defined by
\[ E_{a'} : a'X^{3} + Y^{3} + Z^{3} = 0,\] \[E_{a'^{2}} : a'^{2}X^{3} + Y^{3} + Z^{3} = 0,\] and\[ E_{1}: X^{3} + Y^{3} + Z^{3} = 0\] are the only non-$\FF_{p^{k/3}}$-isomorphic twists of $E$ such that any two of the coefficients $\alpha,\beta,\gamma$ are the same. The curve $E'$ with $\alpha = 1$, $\beta = a'$, and $\gamma = a'^{2}$ is isomorphic over $F_{p^{k/3}}$ to \[E_{1}: X^{3} + Y^{3} + Z^{3} = 0.\]
\end{proof}

Now, if $3|k$, where $k$ is the embedding degree of $E/\FF_{p}$, and $\#E(\FF_{p}) \not\equiv 0 \text{ mod } 3$, then by \cite[Theorem 3]{2006/hess} and the above Lemma, for any $a \in \FF_{p}^{*}$, we have that
\[E_{a}(\FF_{p^{k}}) \cong \bigoplus_{a' \in \FF_{p^{k/3}}^{*}/(\FF_{p^{k/3}}^{*})^{3}} E_{a'}(\FF_{p^{k/3}}).\] 
In particular, every point $Q \in E_{a}(\FF_{p^{k}})$ is the image of a point $Q_{a'} \in E_{a'}(\FF_{p^{k/3}})$, for some $a' \in \FF_{p^{k/3}}^{*}/(\FF_{p^{k/3}}^{*})^{3}$, under the morphism
\[\begin{array}{rccc}
\phi_{a'}:& E_{a'} & \longrightarrow & E \\
& (x:y:z) & \mapsto & (b^{-1}x:y:z),
\end{array}\]
where $b \in \FF_{p^{k}}$ satisfies $b^{3} = a'/a$. Hence, writing $Q = (x_{Q}:y_{Q}:z_{Q})$, we can conclude that only $x_{Q}$ is in the full extension field $\FF_{p^{k}}$, as $y_{Q}$ and $z_{Q} \in \FF_{p^{k/3}}$. 
\\
%To use the twists of curves, the degree $d$ must divide $k$.
%For the case $k \equiv 3 \pmod{18}$ and $k \equiv 9,15 \pmod{18}$,
%the only possible twisted degree is $d = 3$.
%For the case $k \equiv 0 \pmod{6}$ where $18 \nmid k$,
%all twisted degrees are possible, i.e., $d \in \{2,3,4,6\}$.

%Recall that we focus only on curves that both the curves and theirs twists contain points of order 3.

Also, as we are currently bound to using algorithms written for curves given in Weierstrass form, when we have computed a degree 3 twist $E'$ of a given elliptic curve $E$, we have to check whether it can be written in Hessian form, that is, whether or not $E'$ contains a point of order 3. To do this, we use the formulas for the number of points on twisted curves given in~\cite{2006/hess}.
The formulas for calculating the number of points on twisted curves always come in pairs as there are exactly two twists, as we proved above.
For example, the formulas stated in~\cite{2006/hess} are as follows:
\begin{align*}
\# E' (\F{p}) &= p + 1 - (3f - t) / 2 & with & \quad t^2 - 4p = -3f^2, \\
\# E' (\F{p}) &= p + 1 - (-3f- t) / 2 & with & \quad t^2 - 4p = -3f^2.
\end{align*}
To determine the unique twist containing the pre-image of our point $Q \in E(\FF_{p^{k}})$,
we use the fact that $\#E'(\F{p})$ must also be divisible by $r$.
\\
Note that in Constructions 1 and 2, the only twists that occur are of degree 3, even without the restriction that the resulting curve should be in Hessian form. For Construction 3, there exist twists of degrees different from 3, but the resulting curves will not be in Hessian form.
%we found that for all Construction 1, 2 and 3,
%the only possible twist is $d = 3$.

