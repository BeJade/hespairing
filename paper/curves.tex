\section{Curve constructions}
\label{sec:curves}

Even though every elliptic curve can be written in a Weierstrass form,
only those that contain points of order 3 can be written in (twisted) Hessian form.
Almost all methods to generate pairing-friendly curves are for generating
pairing-friendly Weierstrass curves, 
so we find pairing-friendly Hessian curves by
searching through constructions of pairing-friendly Weierstrass curves for curves that have points of order 3,
and converting those curves into Hessian form.
The families that we present below are guaranteed to have points of order 3.

In order to give fast formulas for curve arithmetic, it is desirable for the pairing-friendly curves that we consider to have \emph{twists}.
Recall that a \emph{degree-$d$ twist} of an elliptic curve $E/\F{q}$ is an elliptic curve $E'/\F{q^e}$ that is
isomorphic to $E$ over a degree-$d$ extension of $\F{q^e}$ but not over any smaller field.
Recall also (e.g. \cite{2009/silverman-arithmetic}) that the only degrees of twists that occur for elliptic curves are $d \in \{2,3,4,6\}$ such that $d|k$, and that degree 3 and 6 twists occur only for elliptic curves with $j$-invariant 0.
We concentrate in this article on twists of degree 3, as motivated by our aforementioned interest in embedding degrees $k = 15$ and 21.
Twisted Hessian curves with $j$-invariant zero are of the form
\[\mathcal{H}_a: aX^3 + Y^3 + Z^3 = 0.\]
This motivates our interest in Hessian curves in particular:
Suppose that $a \in \F{q}$ is a non-cube such that
for $\omega \in \F{q^3}$ with $a = \omega^3$, the element
$\omega$ generates $\F{q^k}$ as a $\F{q^{k/3}}$-vector space.
Then $\mathcal{H}_{a}$ is a degree-3 twist of $\mathcal{H}_1$; the two curves are isomorphic via
\begin{equation}\label{twistiso}
\begin{array}{rccc}
\varphi: & \mathcal{H}_{a} & \rightarrow & \mathcal{H}_1 \\
& (X:Y:Z) & \mapsto & (\omega X:Y:Z).
\end{array}
\end{equation}
In particular, if $R' \in \mathcal{H}_{a}(\F{q^{k/3}})$, then
$\varphi(R') \in \G{2}$.
Analogously to \cite{2003/bls}, we choose the $\G{2}$ input point for the pairing
from $\varphi(\mathcal{H}_{a}(\F{q^{k/3}}))$.
The simplicity of the twist isomorphism 
allows us to do many calculations in $\F{q^{k/3}}$ instead of $\F{q^k}$,
as explained in detail on a case-by-case basis in Section \ref{sec:lines}.

\subsection{Degree six twists of Hessian curves}\label{sextic}

In this article we include, for completeness, formulas for computing pairings
of Hessian curves with even embedding degree. 
As we want to make use of the natural twist of degree 3, 
the embedding degrees that we consider are also divisible by 3, 
so that we are in fact considering embedding degrees divisible by 6.

As mentioned above, degree 6 twists only occur for elliptic curves
with $j$-invariant zero. 
Let $a$ and $\omega$ be as in the previous section
and let $\alpha \in \F{q^2}$ generate $\F{q^{k/3}}$ as a 
$\F{q^{k/6}}$-vector space. Then
\[\F{q^k} = \F{q^{k/6}} + \alpha\F{q^{k/6}} 
+\omega\F{q^{k/6}} + \alpha\omega\F{q^{k/6}}
+\omega^2\F{q^{k/6}} + \alpha\omega^2\F{q^{k/6}}.\] 
Define the triangular elliptic curve
\[T/\F{q} : \alpha^2 VW(V + aW) = U^3.\]
Then we can adapt the isomorphism of \cite[Theorem 5.3]{2012/hessian}
to see that $T$ is a degree-2 twist of $\mathcal{H}_a$ via
the isomorphism
\begin{equation}\label{quadtwist}
\begin{array}{rccc}
\psi: & T & \rightarrow & \mathcal{H}_a \\ 
& (U:V:W) & \mapsto &
(U:\beta(\alpha V - 54W) : \beta(-\alpha V + 54\zeta_3^2 W)),
\end{array}
\end{equation}
where $\beta = \zeta_3-\zeta_3^2$ and 
$\zeta_3 \in \F{q}$ is a primitive cube root of unity.
In particular, the triangular elliptic curve $T$ is a degree-6 twist of $\mathcal{H}_1$ via the composition $\varphi \circ \psi$,
where $\varphi$ is as given in \eqref{twistiso}.

\subsection{Checking for points of order 3}\label{twist}

Let $E/\F{q}$ be an elliptic curve. There is a Hessian model of $E$ if and only if $E(\F{q})$ contains a point of order 3. 
To apply the formulas in the following sections we require both $E$ and the 
degree-3 twist of $E$ that we consider to have order 3.
Recall that
\[\#E(\F{q}) = q+1-t,\]
where $t$ is the trace of Frobenius; 
by~\cite{2006/hess} the two non-trivial degree-3 twists $E'$ satisfy:
\begin{align*}
\# E' (\F{q}) &= q + 1 - (3f - t) / 2 & \text{with} & \quad t^2 - 4q = -3f^2, \\
\# E' (\F{q}) &= q + 1 - (-3f- t) / 2 & \text{with} & \quad t^2 - 4q = -3f^2.
\end{align*}
It is also necessary that for the twist $E'$ that we use $\#E'(\F{q})$ is divisible by $r$ (recall that $r$ was the largest prime factor of $\# E(\F{q})$) and exactly one of the two possible twists satisfies this condition.

So to choose a family for which the elliptic curve $E$ can be rewritten in Hessian form together with a degree-3 twist, it suffices to check that 3 divides $q+1-t$ and that $3r$ divides $q + 1 - (\pm 3f - t) / 2$ (for one choice of sign).

%------------------------

%We will see below that if $E$ is an elliptic curve over $\FF_{q}$
%where $q \equiv 1 \text{ mod } 3$ and can be written in the form
%\[E_{a}: aX^{3} + Y^{3} + Z^{3} = 0\]
%then $E_{a}$ has twists of degree 3. We will also see that if the embedding degree $k$ is divisible by 3,
%the existence of these twists allow us to assume without loss of generality that 2 of the 3 coordinates
%of the point $Q \in E(\FF_{q^{k}})$ are in $\FF_{q^{k/3}}$, reducing the cost of the computation.

%\begin{lemma}
%Suppose that
%\[E: aX^{3} + Y^{3} + Z^{3} = 0\] is an elliptic curve defined over a field $\FF_{q}$ with embedding degree $k$,
%that $q \equiv 1 \text{ mod } 3$, and that $3|k$.
%Then up to $\FF_{q^{k/3}}$-isomorphism there are exactly two degree 3 twists of $E$ defined over $\FF_{q^{k/3}}$.
%These twists are given by
%\[E_{a'}: a'X^{3} + Y^{3} + Z^{3} = 0,\] for $a \neq a' \in \FF_{q^{k/3}}^{*}/(\FF_{q^{k/3}}^{*})^{3}$.
%\end{lemma}
%
%\begin{proof}
%Every supersingular elliptic curve with a point of order 3 is of the form
%\[E': \alpha X^{3} + \beta Y^{3} + \gamma Z^{3} = 0,\] where $\alpha,\beta,\gamma \in \FF_{q^{k/3}}$.
%Furthermore, we know that $\alpha$, $\beta$, and $\gamma$ are non-zero as $E'$ is non-singular.
%Hence every twist of $E$ is isomorphic over $\FF_{q^{k/3}}$ to an elliptic curve $E'$ of the above form
%with $\alpha,\beta,\gamma \in \FF_{q^{k/3}}^{*}/(\FF_{q^{k/3}}^{*})^{3}$.
%Let $a'$ be a generator of the group $\FF_{q^{k/3}}^{*}/(\FF_{q^{k/3}}^{*})^{3}$, so that
%\[\FF_{q^{k/3}}^{*}/(\FF_{q^{k/3}}^{*})^{3} = \{1,a',a'^{2}\}.\] The curves defined by
%\[ E_{a'} : a'X^{3} + Y^{3} + Z^{3} = 0,\] \[E_{a'^{2}} : a'^{2}X^{3} + Y^{3} + Z^{3} = 0,\]
%and\[ E_{1}: X^{3} + Y^{3} + Z^{3} = 0\] are the only non-$\FF_{q^{k/3}}$-isomorphic twists of $E$
%such that any two of the coefficients $\alpha,\beta,\gamma$ are the same.
%The curve $E'$ with $\alpha = 1$, $\beta = a'$, and $\gamma = a'^{2}$ is isomorphic over $F_{q^{k/3}}$ to \[E_{1}: X^{3} + Y^{3} + Z^{3} = 0.\]
%\end{proof}
%
%Now, if $3|k$, where $k$ is the embedding degree of $E/\FF_{q}$, and $\#E(\FF_{q}) \not\equiv 0 \text{ mod } 3$,
%then by \cite[Theorem 3]{2006/hess} and the above Lemma, for any $a \in \FF_{q}^{*}$, we have that
%\[E_{a}(\FF_{q^{k}}) \cong \bigoplus_{a' \in \FF_{q^{k/3}}^{*}/(\FF_{q^{k/3}}^{*})^{3}} E_{a'}(\FF_{q^{k/3}}).\] 
%In particular, every point $Q \in E_{a}(\FF_{q^{k}})$ is the image of a point $Q_{a'} \in E_{a'}(\FF_{q^{k/3}})$,
%for some $a' \in \FF_{q^{k/3}}^{*}/(\FF_{q^{k/3}}^{*})^{3}$, under the morphism
%\[\begin{array}{rccc}
%\phi_{a'}:& E_{a'} & \longrightarrow & E \\
%& (x:y:z) & \mapsto & (b^{-1}x:y:z),
%\end{array}\]
%where $b \in \FF_{q^{k}}$ satisfies $b^{3} = a'/a$.
%Hence, writing $Q = (x_{Q}:y_{Q}:z_{Q})$,
%we can conclude that only $x_{Q}$ is in the full extension field $\FF_{q^{k}}$, as $y_{Q}$ and $z_{Q} \in \FF_{q^{k/3}}$. 
%\\

%Also, as we are currently bound to using algorithms written for curves given in Weierstrass form,
%when we have computed a degree 3 twist $E'$ of a given elliptic curve $E$,
%we have to check whether it can be written in Hessian form, that is,
%whether or not $E'$ contains a point of order 3. To do this, we use the formulas for the number of points on twisted curves given in~\cite{2006/hess}.
%The formulas for calculating the number of points on twisted curves always come in pairs as there are exactly two twists, as we proved above.
%For example, the formulas stated in~\cite{2006/hess} are as follows:
%\begin{align*}
%\# E' (\F{p}) &= p + 1 - (3f - t) / 2 & with & \quad t^2 - 4p = -3f^2, \\
%\# E' (\F{p}) &= p + 1 - (-3f- t) / 2 & with & \quad t^2 - 4p = -3f^2.
%\end{align*}
%To determine the unique twist containing the pre-image of our point $Q \in E(\FF_{p^{k}})$,
%we use the fact that $\#E'(\F{p})$ must also be divisible by $r$.
%\\
%Note that in Constructions 1 and 2, the only twists that occur are of degree 3,
%even without the restriction that the resulting curve should be in Hessian form.
%For Construction 3, there exist twists of degrees different from 3, but the resulting curves will not be in Hessian form.
%%we found that for all Construction 1, 2 and 3,
%%the only possible twist is $d = 3$.

%%%%%%%%%%%%%%%%%%%%%%%%%%%%%%%%%%%%%%%%%%%%%%%%%%%%%%%%%%%%%%%%%%%%%%%%%%%%%%%%%%%%%%%%%%%%%%%%%%%%%%%%%%%%%%%%%%%%%%%%

\subsection{Generating curves}
\label{subsec:gencurves}

Recall that $E$ is an elliptic curve defined over a finite field $\F{q}$ where $q$ is prime,
and $r$ is the largest prime factor of $\# E(\F{q})$.
The embedding degree $k$ is the smallest integer $k$ such that $r | q^k -1$.
Constructions of parametric families of pairing-friendly curves give an elliptic curve $E$ with integral coeffiients and polynomials $q(x)$ and $r(x)$,
where for each $x_0$ such that $q(x_0)$ is prime and $r(x_0)$ has a large prime factor, the reduction of $E$ mod $q(x_0)$ is a pairing-friendly curve with parameters $q = q(x_0)$ and $r = r(x_0)$.

Cyclotomic families are families of curves where the underlying field $K$ is a cyclotomic field,
the size $r$ of the largest prime-order subgroup of the group of $\F{q}$-points is a cyclotomic polynomial,
and the field $K$ contains $\sqrt{-D}$ for some small discriminant $D$.
We searched through~\cite{2010/freeman} and found three cyclotomic-family constructions that satisfy the conditions outlined in the previous section.
These constructions are based on a cyclotomic field containing a cube root of unity,
i.e., fields contain $\sqrt{-3}$.
Therefore, we choose the discriminant $D = 3$.

The following constructions generate pairing-friendly Weierstrass curves
which have a (twisted) Hessian model \cite[Section 5]{2015/hessian}.
Note that twists of these curves (see Section~\ref{twist}) are also expressible in twisted Hessian form.
%We categorized these constructions by embedding degrees.
%Constructions 1 and 2 are for embedding degree divisible by 3, but not 6;
%our analysis shows that for embedding degree divisible by 6 there are 
%other choices of curve shapes that are preferable to Hessian form
%(with respect to the efficiency of the pairing implementation), 
%so these are the most relevant constructions for this paper.
%Construction 3 is for even embedding degree divisible by 6 which
%we include only for completeness.
%Note that $q(x)$ is a prime,
%$r(x)$ is the size of a subgroup of $E(\F{q(x)})$ (which may not have prime order),
%and $t(x)$ is a trace of Frobenius of $E/\F{q(x)}$.
We denote the cyclotomic polynomial of degree $n$ by $\Phi_{n}(x)$ .

%Recall that in Section~\ref{sec:Tate}, we defined
%\begin{itemize}
%\item $p$ to be prime,
%\item $E$ to be an elliptic curve over $\FF_{p}$,
%\item $r$ to be the size of the largest prime order subgroup of $E(\FF_{p})$, and
%\item $k$ to be smallest integer for which $r | p^{k}-1$.
%\end{itemize}
%Under these assumptions, we also defined 
%\[\begin{array}{ccc}
%\mathbb{G}_{1} = E[r] \cap \ker(\phi_{p} - [1]) & \text{and} & \mathbb{G}_{2} = E[r] \cap \ker(\phi_{p} - [p]) \subseteq E(\FF_{p}^{k}),
%\end{array}\]
%and the reduced Tate pairing restricted to $\mathbb{G}_{1} \times \mathbb{G}_{2}$ was given by
%\[\begin{array}{rccc}
%e_{r}:& \mathbb{G}_{1} \times \mathbb{G}_{2} & \longrightarrow & \mu_{r} \\
%& (P,Q) & \mapsto & f_{r,P}(Q)^{\frac{p^{k}-1}{r}}.
%\end{array}\]
%We will refer to an elliptic curve $E/\FF_{p}$ where $p,r,k$ are known and satisfy the above properties as \emph{pairing-friendly}.
%To construct the reduced Tate pairing, we search for pairing-friendly elliptic curves with a point of order 3, as these can be written in Hessian form.
%There are many constructions of parametric families of pairing-friendly curves listed in \cite{2010/freeman}.
%We recall below those families for which the curves also have a point of order 3, as shown in \cite[Section 5]{2015/hessian}.

%%%%%%%%%%%%%%%%%%%%%%%%%%%%%%%%%%%%%%%%%%%%%%%%%%%%%%%%%%%%%%%%%%%%%%%%%%%%%%%%%%%%%%%%%%%%%%%%%%%%

\subsubsection{Construction 1: $k \equiv 3 \pmod{18}$.}
\label{con1}

This construction follows {Construction 6.6} in~\cite{2010/freeman}.
Pairing-friendly curves with embedding degree $k \equiv 3 \pmod{18}$
can be constructed using the following polynomials:
\begin{align*}
r(x) &= \Phi_{2k}(x),	\\
t(x) &= x^{k/3+1} + 1,	\\
q(x) &= \frac{1}{3} (x^2 - x + 1) (x^{2k/3} - x^{k/3} + 1) + x^{k/3+1}.
\end{align*}
For this construction,
the resulting curves and their twists all have points of order 3.
However, there is no such $x_0$ for which
both $q(x_0)$ and $r(x_0)$ are prime.
This means that $r(x_0)$ factors, and the largest prime-order subgroup of $E(\FF_{q})$ actually has less than $r(x_0)$ elements.
Recall that the discriminant $D = 3$. 
This implies in particular that the curves are defined by an equation of the form $y^2 = x^3 + b$.
The only possible twists are cubic ($d=3$) twists.
The $\rho$-value of this family is $\rho = (2k/3 + 2)/\varphi(k)$ where $\varphi$ is the Euler $\varphi$-function.
For $k = 21$ this gives $\rho = 4/3$.

%%%%%%%%%%%%%%%%%%%%%%%%%%%%%%%%%%%%%%%%%%%%%%%%%%

\subsubsection{Construction 2: $k \equiv 9,15 \pmod{18}$.}
\label{con2}

This construction follows {Construction 6.6} in~\cite{2010/freeman}.
Pairing-friendly curves with embedding degree $k \equiv 9,15 \pmod{18}$
can be constructed using the following polynomials:
\begin{align*}
r(x) &= \Phi_{2k}(x),	\\
t(x) &= -x^{k/3+1} + x + 1,	\\
q(x) &= \frac{1}{3} (x+1)^2 (x^{2k/3} - x^{k/3} + 1) - x^{2k/3+1}.
\end{align*}
This satisfies all the same properties as Construction 1.
For $k=15$ the $\rho$-value is $\rho=3/2$.

%%%%%%%%%%%%%%%%%%%%%%%%%%%%%%%%%%%%%%%%%%%%%%%%%%

\subsubsection{Construction 3: $k \equiv 0 \pmod{6}$ and $18 \nmid k$.}
\label{con3}

This construction follows {Construction 6.6} in~\cite{2010/freeman}.
Pairing-friendly curves with embedding degree $k \equiv 0 \pmod{6}$ where $18 \nmid k$
can be constructed using the following polynomials:
\begin{align*}
r(x) &= \Phi_k(x),	\\
t(x) &= x+1,		\\
q(x) &= \frac{1}{3} (x-1)^2 (x^{k/3} - x^{k/6} + 1) + x.
\end{align*}
For this construction,
the resulting curves and their twists all have points of order 3.
There also exists $x_0$ such that both $q(x_0)$ and $r(x_0)$ are prime.
The curves generated by this construction admit sextic twists.
The $\rho$-value for this construction is given by $\rho = (k/3+2)/\varphi(k)$
where $\varphi$ is the Euler $\varphi$-function.
For $k=12$ this gives $\rho = 3/2$ and for $k=24$ this gives $\rho = 5/4$.

%%%%%%%%%%%%%%%%%%%%%%%%%%%%%%%%%%%%%%%%%%%%%%%%%%

%\subsubsection{Sage scripts.}
%We provide Sage scripts (Figure~\ref{fig:sscript})
%for constructing pairing-friendly Weierstrass curves with embedding degree $k = 21$
%(see Construction 1 above)
%which can be converted to twisted Hessian curves.
%Note that the scripts also work for different embedding degrees $k \equiv 3 \pmod{18}$ 
%by changing \texttt{k = 21} to the desired embedding degrees.
%For other constructions,
%in addition to changing the embedding degree,
%the polynomials \texttt{rx, r, t, q} also need to be replaced by the corresponding ones.
%For auxiliary functions in \texttt{util.sage}, please refer to Appendix~\ref{app:sage}.
%
%\begin{figure}
%\hrule\medskip
%\setstretch{0.8}
%\VerbatimInput[fontsize=\scriptsize,commandchars=\\\{\},tabsize=2]{./../code/revise21.sage}
%\setstretch{1}
%\hrule
%\caption{Sage scripts to generate pairing-friendly Weierstrass curves of embedding degree $k = 21$
%that can be converted into twisted Hessian form and allow twist of degree 3.}
%\label{fig:sscript}
%\end{figure}
%
%Figure~\ref{fig:sout} shows the output of
%pairing-friendly Weierstrass curves generated by the Sage scripts in Figure~\ref{fig:sscript}.
%Note that the purpose of these scripts is to provide a concrete method to generate
%pairing-friendly Weierstrass curves that can be converted into the twisted Hessian form,
%and not to propose a secure curve at any particular security levels.
%We will later on use the parameters in the output from the scripts for our toy example.
%Therefore, we use rather small prime field size for the purpose of checking correctness of the algorithm.
%
%\begin{figure}
%\hrule\medskip
%{\scriptsize
%\begin{verbatim}
%x = 5054
%t = 425678681440265235217560699137
%q = 60388831224640627688578323697279079263669799534119323634669
%r = 277784988873145112452421916846435035271854071
%a,b = 0 144
%\end{verbatim}
%}
%\hrule
%\caption{The output of the Sage scripts in Figure~\ref{fig:sscript}
%to generate pairing-friendly Weierstrass curves where
%`x' denotes an integer that results in $q(x) = \frac{1}{3} (x^2 - x + 1) (x^{2k/3} - x^{k/3} + 1) + x^{k/3+1}$ being prime;
%`t' denotes a trace of Frobenius;
%`q' denotes a prime field;
%`r' denotes a subgroup of the resulting curve; and
%`a,b' denotes the parameters of the Weierstrass curve $y^2 = x^3 + ax + b$.
%}
%\label{fig:sout}
%\end{figure}

%%%%%%%%%%%%%%%%%%%%%%%%%%%%%%%%%%%%%%%%%%%%%%%%%%%%%%%%%%%%%%%%%%%%%%%%%%%%%%%%%%%%%%%%%%%%%%%%%%%%


%\subsection{Converting Weierstrass to twisted Hessian curves}
%\label{subsec:w2h}

For all the constructions outlined above,
the curves are given in Weierstrass form as
\[v^2 = u^3 + b.\]
To convert a pairing-friendly Weierstrass curve of the above form
that has a point $(u_3,v_3)$ of order 3
into twisted Hessian form, we refer to~\cite{2015/hessian}.
%Specifically, the proofs of Theorem 5.2 and {Theorem 5.3} in~\cite{2015/hessian} describe the isomorphism from
%Weierstrass curves to twisted Hessian curves via triangular curves using a series of substitutions.
The authors give explicit transformations showing that there is a Hessian model of the above curve given by
\[aX^3 + Y^3 + Z^3 = 0,\]
where
$a = 27(u_3^6/v_3^3-2v_3)$.
%To be more precise, the proofs start by giving a point of order 3 $P_3 = (u_3,v_3)$ and
%writing a curve (defined over $\F{q}$) in a long Weierstrass form
%$$ v^2 + e_1 uv + e_3 v = u^3 + e_2 u^2 + e_4 u + e_6. $$
%Note however that in our case, we consider curves of the form $y^2 = x^3 + ax + b$.
%Thus, the curve parameters $e_1 = e_3 = e_2 = e_4 = 0$, and $e_6 = b$.
%Therefore, we start with the equation
%$$v^2 = u^3 + b.$$
%
%Next step is to substitute $u = x + u_3$ and $v = t + v_3$
%to obtain the curve of the form
%$$ t^2 + c_1 xt + c_3 t = x^3 + c_2 x^2 + c_4 x + c_6. $$
%In our case, we have
%$$ t^2 + (2v_3)t = x^3 + (3u_3)x^2 + (3u_3^2)x + (b-v_3^2). $$
%This means that
%$$ c_3 = 2v_3, \qquad c_2 = 3u_3, \qquad c_4 = 3u_3^2, \qquad c_6 = b-v_3^2. $$
%
%Then, another substitution $t = y + \lambda x$ for $\lambda \in \F{q}$ is performed,
%which leads to a curve of the form
%$$ y^2 + a_1 xy + a_3 y = x^3 + a_2 x^2 + a_4 x + a_6, $$
%where $a_2 = a_4 = a_6 = 0$ and hence in triangular form
%$y^2 + dxy + ay = x^3$.
%In our case, we obtain
%$$ y^2 + (2\lambda)xy + c_3 y = x^3 + (c_2 - \lambda^2)x^2 + (c_4 - c_3 \lambda)x + c_6. $$
%This means that
%\begin{gather*}
%a_1 = 2\lambda, \qquad a_3 = c_3, \\
%a_2 = c_2 - \lambda^2 = 0, \qquad a_4 = c_4 - c_3 \lambda = 0, \qquad a_6 = c_6 = 0,
%\end{gather*}
%which implies $\lambda = c_4 / c_3$.
%
%We rename the parameters to $d = a_1$ and $a = a_3$.
%The curve parameters $d'$ and $a'$ for twisted Hessian curve
%$$ H: a'X^3 + Y^3 + Z^3 = d'XYZ$$
%are defined as
%$$ a' = d^3 - 27a, \qquad d' = 3d. $$
%
%By making the substitution $x = U/W$ and $y = V/W$,
%we homogenize the above form of the triangular curve $y^2 + dxy + ay = x^3$ to
%$$ VW(V + dU + aW) = U^3. $$
%
%Once we obtain the parameters $a'$ and $d'$ for twisted Hessian curve,
%it remains to map the coordinates $(U : V : W)$ of the triangular curve
%to coordinates $(X : Y : Z)$ of the twisted Hessian curve.
%This map $\phi(U:V:W) = (X:Y:Z)$ is defined as:
%\begin{align*}
%X &= U, \\
%Y &= \omega(V + dU + AW) - \omega^2 V - aW, \\
%Z &= \omega^2 (V + dU + aW) - \omega V - aW,
%\end{align*}
%where $\omega \in \F{q}$, $\omega^3 = 1$ and $\omega \ne 1$.
%
%To summarize, to compute related parameters (before the map $\phi$)
%we need to perform the following computation:
%\begin{align*}
%c_3 &= 2 v_3,	&	c_4 &= 3 u_3^2,		&	\lambda &= c_3 / c_4.	\\
%d &= 2 \lambda,	&	a &= c_3.		&	\\
%d' &= 3 d,	&	a' &= d^3 - 27 a.	&
%\end{align*}
%
%Next,
%given a point $(U:V:W)$, a point of order 3 $(u_3,v_3,1)$, and related parameters
%$a,d,\lambda,\omega$,
%we compute:
%\begin{gather*}
%U = U - u_3, \qquad
%V = V - v_3 - \lambda (U - u_3), \qquad
%W = W, \\
%A = \omega(V + dU + aW),	\qquad
%B = \omega V,	\qquad
%C = aW.
%\end{gather*}
%Then, the map $\phi$ is computed as:
%\begin{align*}
%X &= U,	\\
%Y &= \omega(V + dU + aW) - \omega^2V - aW = A - \omega B - C,	\\
%Z &= \omega^2(V + dU + aW) - \omega V - aW = \omega A - B - C.
%\end{align*}
%
%Note that in our case, the parameter $d'$ is zero.
%To convert the coordinates, we also need to compute $\omega \in \F{q}$
%where $\omega^3 = 1$ and $\omega \ne 1$.
Let $\mul, \sqr$ and $\mulp{c}$ denote
field multiplication, field squaring and field multiplication by a small constant respectively.
%Therefore, to compute related parameters (before the map $\phi$)
%this step costs $\mul + 2\sqr + 5\mulp{c}$
%plus one inversion for $\lambda = c_4 / c_3$
%and one cube root computation for computing $\omega$.
%The cost for computing the map $\phi$ is $8\mul$.
They compute the total cost for the whole conversion to be $9\mul + 2\sqr + 5\mulp{c}$
plus one inversion and one cube root computation.


%\subsubsection{Sage scripts.}
%We provide Sage scripts (Figure~\ref{fig:scon})
%for converting Weierstrass curves into twisted Hessian form.
%The function \texttt{w2h} in the scripts takes as inputs
%the prime field \texttt{fq} and
%Weierstrass curve parameters \texttt{a} and \texttt{b}.
%It computes the necessary parameters for converting the Weierstrass curve
%with the given input to the corresponding twisted Hessian curve.
%For auxiliary functions in \texttt{util.sage}, please refer to Appendix~\ref{app:sage}.
%
%Specifically, the function first computes a point of order 3 $P_3 = (u_3,v_3)$.
%This is done by randomly choosing a point on the curve,
%checking that it is not the neutral element,
%and multiplying by a correct cofactor.
%Next, it computes parameters for converting curves in Weierstrass form to triangular form,
%namely, \texttt{c3}, \texttt{c4}, \texttt{lambda}, \texttt{a1}, \texttt{a3}, \texttt{td} and \texttt{ta}
%which correspond to the above mentioned $c_3, c_4, \lambda, d, a, d'$ and $a'$ respectively.
%Then, it computes parameters for converting curves in triangular form to twisted Hessian form,
%namely, \texttt{ha} and \texttt{hd} which correspond to the above mentioned $a'$ and $d'$.
%Finally, it computes \texttt{omega} which is a cube root element in $\F{q}$.
%The function then outputs:
%the coordinates of a point of order 3 $P_3 = (u_3,v_3)$;
%parameters $d$ and $a$ for the conversion;
%twisted Hessian curve parameters $a'$ and $d'$;
%and a cube root element $\omega$.
%
%
%\begin{figure}
%\hrule\medskip
%\setstretch{0.8}
%\VerbatimInput[fontsize=\scriptsize,commandchars=\\\{\},tabsize=2]{./../code/convert.sage}
%\setstretch{1}
%\hrule
%\caption{Sage scripts to convert Weierstrass curves $E/\F{q}: y^2 = x^3 + ax + b$
%into twisted Hessian form $a'X^3 + Y^3 + Z^3 = d'XYZ$.}
%\label{fig:scon}
%\end{figure}
%
%Figure~\ref{fig:conout} shows an example of the output of the function \texttt{w2h} of Sage scripts shown in Figure~\ref{fig:scon}
%which uses the output of the previous Sage scripts in Figure~\ref{fig:sscript} for input $r,q,a,b$,
%i.e., a pairing-friendly Weierstrass curve $E/\F{q}: y^2 = x^3 + ax + b$ with a subgroup of prime order $r$.
%Note that we use \texttt{p3 = E.random\_element()} in our scripts.
%This means that it is possible that the scripts give different output when run multiple times.
%
%\begin{figure}
%\hrule\medskip
%{\scriptsize
%\begin{verbatim}
%u3 = 0
%v3 = 12
%d  = 0
%a  = 24
%a' = 60388831224640627688578323697279079263669799534119323634021
%d' = 0
%w  = 17923080803972475283541924324100117212007204172538782666
%\end{verbatim}
%}
%\hrule
%\caption{The output of the Sage scripts in Figure~\ref{fig:scon}
%to convert Weierstrass curve into twisted Hessian form where
%`u3,v3' denotes point-of-order-3 $P_3 = (u_3,v_3)$;
%`d' and `a' denote the parameters $d$ and $a$ which are used for the conversion;
%`a$'$' and `d$'$' denote parameters for twisted Hessian curve $a'X^3 + Y^3 + Z^3 = d'XYZ$; and
%`w' denotes a cube root element $\omega$.
%}
%\label{fig:conout}
%\end{figure}

